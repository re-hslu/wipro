% ================== Kapitel: Problem, Fragestellung, Vision ==================

\section{Ausgangslage und Problemstellung}
Die HSLU-Informatik betreibt eigene Mobile-Apps für die Departemente
Informatik sowie Technik \& Architektur auf den Plattformen \textit{iOS} und \textit{Android}.
Im Rahmen einer Forschungsarbeit wurde die Codebasis für iOS stark erweitert; 
diese Änderungen müssen nun in die Android-Codebasis auf Basis von Jetpack Compose übernommen werden.

Das bestehende Android-Projekt basiert noch auf XML-Layouts und ist nicht
an das aktuelle Backend-API angebunden.
Dies soll im Rahmen von AI-First-Techniken erfolgen.
In diesem Zusammenhang sollen verschiedene Möglichkeiten, Ansätze und Tools evaluiert werden.
Die passendste Technologie wird daraufhin für die konkrete Umsetzung verwendet.

\section{Ziel der Arbeit und erwartete Resultate}
Ziel:  
Eine vollständig funktionsfähige Jetpack-Compose-App soll im Google Play Store veröffentlicht werden.  
Die Umsetzung erfolgt als Single Codebase mit Multi-Feature- und Multi-Tenant-Unterstützung unter Anwendung von Domain-Driven Design (DDD).

\vspace{0.3cm}
\noindent
Erwartete Resultate:
\begin{itemize}
    \item Projektmanagement-Artefakte:  
    Statusberichte, Projektplan inklusive fortlaufender Risikoanalyse, Issues/Epics und Stakeholderübersicht.
    \item App-Artefakte:
    \begin{itemize}
        \item Evaluation und Auswahl geeigneter AI-Ansätze und Tools für Mobile Engineering.
        \item Implementierung der fehlenden Features (bzw. Neuaufbau) in Jetpack Compose mit Tests und Dokumentation für aktuelle targetSDKs.
        \item Release des neuesten Entwicklungsstands via GitLab CI/CD und Fastlane in den Google Play Store.
    \end{itemize}
    \item HSLU-Artefakte:  
    Projektdokumentation gemäss HSLU-Standards sowie Zwischen- und Schlusspräsentation.
\end{itemize}