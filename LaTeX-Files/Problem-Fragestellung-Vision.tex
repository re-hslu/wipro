\section{Ausgangslage und Problemstellung}\vspace{-1.5em}

Die Hochschule Luzern betreibt für die Departemente Informatik sowie Technik \& Architektur eigene mobile
Applikationen auf den Plattformen iOS und Android. Während die iOS-Applikation in den vergangenen
Semestern weiterentwickelt wurde,
befindet sich die Android-Version aktuell in einem teilweise veralteten Zustand.

Auf Android existiert einerseits eine veröffentlichte Applikation auf Basis klassischer XML-Layouts, welche
weder an das aktuelle Backend-API angebunden ist noch alle Features der iOS-Version enthält. 
Andererseits liegt eine nicht veröffentlichte Jetpack-Compose-Applikation vor, die zwar
bereits auf einem deklarativen UI-Ansatz basiert, jedoch ebenfalls nicht mehr dem aktuellen Stand der
Technik entspricht und funktional hinter der iOS-Version zurückbleibt.

Diese Situation führt dazu, dass der Entwicklungsstand der Android-Applikationen weit hinter der iOS-Applikationen zurückliegt. 
Eine konsistente App-Landschaft über beide Plattformen hinweg ist nicht gegeben. Gleichzeitig besteht der
Anspruch, neue Features, die auf iOS bereits umgesetzt wurden, auch auf Android bereitzustellen.
Vor diesem Hintergrund ergibt sich das Bedürfnis, die bestehende Android-Jetpack-Compose-
Applikation grundlegend zu modernisieren und funktional an die iOS-Applikationen anzugleichen. 

\section{Ziel der Arbeit und erwartete Resultate}\vspace{-1.5em}

Ziel dieses Projekts ist die Umsetzung einer modernen
Android-Applikation auf Basis von Jetpack Compose, welche an die bestehende
iOS-App der Hochschule Luzern angeglichen ist. Die Applikation soll aber auch Android-
Technologien und Architekturkonzepte nutzen, wartbar und erweiterbar aufgebaut sein,
um dann produktiv über den Google Play Store veröffentlicht werden zu können.

Die Umsetzung soll als modulare Single-Codebase mit Unterstützung für mehrere
Features und Mandanten (Multi-Tenant) erfolgen. Die Architektur soll sich
an der bestehenden iOS-Applikation orientieren, um eine möglichst einheitliche
Plattformstruktur zu erreichen, damit künftige Arbeiten auf beiden Plattformen einfacher zu erledigen sind.

Ein weiterer zentraler Aspekt der Arbeit soll der Einsatz eines AI-First-Ansatzes sein. Ziel ist es,
verschiedene KI-gestützte Entwicklungswerkzeuge im Kontext der mobilen
App-Entwicklung zu evaluieren und deren Einsatzmöglichkeiten, Grenzen und Mehrwerte
praktisch zu untersuchen. Die gewonnenen Erkenntnisse sollen dokumentiert und
reflektiert werden. (Die vollständige Aufgabenstellung ist zu finden im Anhang: ~\ref{sec:aufgabenstellung}).

Die erwarteten Resultate lassen sich wie folgt zusammenfassen:

\begin{itemize}
    \item \textbf{App-Artefakte:}
    \begin{itemize}
        \item Eine produktiv einsetzbare Android-Applikation auf Basis von Jetpack Compose
        \item Umsetzung und Integration der fehlenden bzw. neu aufgebauten Features
        \item Anbindung an das aktuelle Backend-API
        \item Unterstützung aktueller Android-SDKs und Geräte
        \item Automatisierter Build- und Release-Prozess mittels GitLab CI/CD und Fastlane
    \end{itemize}

    \item \textbf{Projektmanagement-Artefakte:}
    \begin{itemize}
        \item Sprint-basierte Planung und Umsetzung mit dokumentierten Issues und Epics
        \item Regelmässige Statusberichte und Sitzungsprotokolle
        \item Fortlaufende Risikoanalyse und Meilensteinübersicht
    \end{itemize}

    \item \textbf{Dokumentation und Evaluation:}
    \begin{itemize}
        \item Projektdokumentation gemäss HSLU-Standards
        \item Evaluation und Reflexion des AI-First-Ansatzes im Entwicklungsprozess
        \item Zwischen- und Schlusspräsentation der Projektergebnisse
    \end{itemize}
\end{itemize}