Während unseres Projektes haben wir unsere Arbeitszeit konsequent in den GitLab-Issues erfasst.  
Um nachvollziehen zu können, wie viel Zeit insgesamt investiert wurde (ohne diese regelmässig manuell zusammenzuzählen) haben wir ein kurzes Skript geschrieben, das die erfassten Zeiten über die GitLab-API ausliest und aggregiert.

Das Ergebnis ist für uns besonders spannend, da es einen guten Überblick darüber gibt, in welchen Bereichen besonders viel Zeit investiert wurde. Die Auswertung ist jedoch nicht vollständig repräsentativ.  
So wurde der Aufwand für wiederverwendbare Features initial dem jeweiligen Hauptticket (z.\,B. \emph{News}) zugeordnet. Diese Funktionalitäten konnten später auch in anderen Bereichen (z.\,B. \emph{Blog}) genutzt werden, wodurch dort ein geringerer Zeitaufwand ausgewiesen ist.

Auch bei der Dokumentation zeigt sich ein plausibles Bild: Wir haben pro Seite mit ungefähr einer Stunde für saubere Dokumentation gerechnet, was sich in der Auswertung in etwa bestätigt.

\begin{longtable}{l r}
\toprule
\textbf{Feature/Arbeit} & \textbf{Zeit (h)} \\
\midrule
\endfirsthead

\toprule
\textbf{Ticket} & \textbf{Zeit (h)} \\
\midrule
\endhead

API-Versionierung                             & 5.00 \\
JSON-View News/Blog/Mensa                     & 9.50 \\
Bootstrapping                                 & 16.00 \\
Navigation (Setting)                          & 7.00 \\
Zwischenpräsentation vorbereiten und halten   & 9.00 \\
Stundenplan / Timetable                       & 11.00 \\
Info / About                                  & 4.00 \\
Events / Events                               & 6.00 \\
CI/CD, Fastlane                               & 11.00 \\
Dokumentation                                 & 73.00 \\
Allgemein / Diverses                          & 2.00 \\
Projektplanung                                & 20.75 \\
Meetings                                      & 35.50 \\
Aufgabenstellung überarbeiten und signieren   & 2.00 \\
Evaluation AI Tools                           & 15.00 \\
News / News                                   & 2.00 \\
Links / Links                                 & 2.00 \\
Mensa / Canteen                               & 2.00 \\
Raumsuche / Roomsearch                        & 34.00 \\
Blog / Blog                                   & 13.00 \\
Error-Bildschirme                             & 5.00 \\
App-Layout, Menüstruktur                      & 7.00 \\
Multi-Tenant Projektstruktur                  & 12.00 \\
Teststruktur                                  & 3.00 \\
Lokaler Speicher (Room)                       & 20.00 \\
Lokale App-Einstellungen (Einstellungen / Settings) & 8.00 \\
Netzwerkdienst                                & 15.00 \\
Lokalisierung                                 & 4.00 \\
(Geplant: Abschlusspräsentation)              & 8.00 \\
\midrule
\textbf{Gesamtzeit}                           & \textbf{361.75} \\
\bottomrule
\end{longtable}

\newpage
\subsection{Aufteilung der Dokumentation}

Die Dokumentation wurde aufgeteilt und jeweils von einer Person erfasst
sowie von der jeweils anderen Person überprüft.

Die gesamte Dokumentation ist in LaTeX geschrieben und befindet sich öffentlich unter:
\url{https://github.com/re-hslu/wipro}

\begin{itemize}
  \item Problemstellung, Fragestellung, Vision: Raphael Eiholzer
  \item Stand der Praxis und Technik: Samuel Kurmann
  \item Ideen und Konzepte: Samuel Kurmann
  \item Methoden: Raphael Eiholzer
  \item Realisierung:
    \begin{itemize}
      \item Applikationsarchitektur und Konfiguration: Raphael Eiholzer
      \item Common-Komponenten: Samuel Kurmann
      \item Features (ausser Roomsearch): Raphael Eiholzer
      \item Feature Roomsearch: Samuel Kurmann
    \end{itemize}
  \item Validierung:
    \begin{itemize}
      \item Validierung: Raphael Eiholzer
      \item Evaluation AI-Tools: Samuel Kurmann
    \end{itemize}
  \item Ausblick: Raphael Eiholzer
\end{itemize}
\newpage