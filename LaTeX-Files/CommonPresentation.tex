\subsection{CommonPresentation}

Das Modul CommonPresentation stellt eine Sammlung wiederverwendbarer UI-Komponenten für Jetpack Compose bereit, 
die von allen Feature-Modulen verwendet werden können. Es bietet standardisierte Komponenten für Buttons, Textfelder, 
Progress-Indikatoren, Dialoge, Fehleranzeigen und weitere UI-Elemente, die eine konsistente Benutzeroberfläche über die gesamte Anwendung hinweg gewährleisten.

\subsubsection*{Ziel und Motivation}

Das Hauptziel besteht darin, Code-Duplikation zu vermeiden und eine konsistente Benutzeroberfläche zu schaffen, 
indem gemeinsame UI-Komponenten zentralisiert werden. Durch die Verwendung von wiederverwendbaren Komponenten wird sichergestellt, 
dass alle Feature-Module das gleiche Design-System verwenden und Änderungen am Design zentral vorgenommen werden können.

\subsubsection*{Umsetzung / Funktionsweise}

Die Komponenten sind als Composable-Funktionen implementiert und folgen den Jetpack Compose Best Practices. 

Die wichtigsten Elemente aus dem \texttt{CommonPresentation}-Modul sind:

\begin{itemize}
    \item \textbf{GenericErrorView}:  
    Eine einfache und zentrale Fehleranzeige, die eine Fehlermeldung
    übersichtlich darstellt. Optional kann ein \glqq Retry\grqq{}-Button
    angezeigt werden, mit dem der fehlgeschlagene Vorgang erneut ausgelöst
    werden kann. Diese Komponente wird in mehreren Features verwendet, um
    Fehler einheitlich darzustellen.
    \hspace{0.15cm}
    \item \textbf{WebViewScreen}:  
    Eine Composable-Komponente zur Anzeige externer Inhalte innerhalb der App.
    Sie stellt eine Android WebView bereit und unterstützt unter anderem
    JavaScript, Cookies und DOM Storage. Je nach Konfiguration können Inhalte
    entweder direkt in der App oder extern im Systembrowser geöffnet werden.
    \hspace{0.15cm}
    \item \textbf{LogoSplashView}:  
    Eine animierte Splash-Screen-Komponente, die beim Start der App ein Logo
    sowie einen Untertitel anzeigt. Nach Abschluss der Animation kann ein
    Callback ausgelöst werden, um automatisch zur nächsten Ansicht zu
    wechseln. Die Komponente wird während des App-Bootstrappings für den
    initialen Startbildschirm verwendet.
    \hspace{0.15cm}
    \item \textbf{CommonBootstrappingProgressView}:  
    Eine wiederverwendbare Composable-Komponente, die den \newline Initialisierungs-
    und Ladeprozess von Modulen kapselt. Sie übernimmt automatisch die
    Ausführung von \newline \texttt{setup()} und \texttt{process()} und stellt je nach
    Status eine Ladeansicht, eine Erfolgsansicht oder eine Fehleransicht dar.
    Dadurch kann der Bootstrapping-Prozess in allen Features einheitlich und
    ohne doppelten Code umgesetzt werden.
\end{itemize}

\hspace{0.05cm}
\begin{lstlisting}[language=Kotlin, caption={Klasse CommonBootstrappingProgressView}]
@Composable
fun CommonBootstrappingProgressView(
    loader: ModuleLoaderContract<*>,
    successView: @Composable () -> Unit
) {
    val status by loader.loadingStatus.collectAsState()
    // ...
}
\end{lstlisting}