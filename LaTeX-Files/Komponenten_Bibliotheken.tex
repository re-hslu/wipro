\subsection{Komponenten und Bibliotheken}

Zu Beginn des Projekts wurde der bestehende Build-Prozess aus dem früheren \texttt{android-xml-multitenant}-Projekt analysiert und modernisiert.  
In der älteren Variante wurden noch veraltete SDKs und Bibliotheken eingesetzt, die nicht mehr dem aktuellen Stand der Android-Entwicklung entsprachen.  
Im Zuge der Neuumsetzung mit Jetpack Compose wurde daher der gesamte Build auf eine aktuelle Versionen der Bibliotheken umgestellt.  

Dies umfasste die Aktualisierung auf ein neues Android-SDK-Level, die Migration auf eine moderne Kotlin-Version (\texttt{2.1.21}) sowie die Aktualisierung der zentralen Bibliotheken.  

Im Folgenden werden die wichtigsten Plugins, SDK-Einstellungen und externen Abhängigkeiten beschrieben, die die Grundlage der neuen Projektstruktur bilden.

\subsubsection{Build Plugins}
Im Projekt werden folgende Gradle-Plugins verwendet:

\begin{itemize}
    \item \textbf{com.android.application}: Basisplugin für Android-Apps.  
    \item \textbf{org.jetbrains.kotlin.android}: Aktiviert Kotlin-Unterstützung für Android.  
    \item \textbf{org.jetbrains.kotlin.plugin.compose}: Integration von Jetpack Compose in Kotlin.  
    \item \textbf{com.google.devtools.ksp}: Kotlin Symbol Processing (KSP) für Codegenerierung (z.~B. Hilt, Room).  
    \item \textbf{dagger.hilt.android.plugin}: Aktiviert Dependency Injection mit Hilt.
\end{itemize}

Diese Plugins stellen die Grundlage für Build-Prozess, Codegenerierung und Abhängigkeitsverwaltung dar.  
Zusätzliche Projekt-Module (\texttt{:common}, \texttt{:features}) werden über \texttt{implementation project(...)} eingebunden, um eine klare Trennung zwischen gemeinsamem Code und Feature-spezifischer Logik zu ermöglichen.

\subsubsection{SDK-Konfiguration}
Das Projekt wurde im Zuge der Modernisierung auf die aktuellste Android-SDK-Version angehoben, um die neuesten Funktionen, Sicherheitsmechanismen und Performance-Verbesserungen der Plattform zu nutzen.  
Seit dem Release von \textbf{Android~16 (SDK~36, Juni~2025)} verwendet das Projekt sowohl die \texttt{compileSdk} als auch die \texttt{targetSdk} in Version~36.  
Dadurch ist sichergestellt, dass die App alle modernen APIs nutzt und gleichzeitig mit den aktuellen Play-Store-Richtlinien kompatibel bleibt.

\begin{itemize}
    \item \textbf{compileSdk}: 36  
    \item \textbf{targetSdk}: 36  
    \item \textbf{minSdk}: 30
\end{itemize}

Mit der \texttt{minSdk}-Version~30 wird weiterhin Android~11 (veröffentlicht im Jahr~2020) unterstützt, was eine breite Geräteabdeckung sicherstellt.  
Damit ist die App auch auf Geräten lauffähig, die bis zu fünf Jahre alt sind, und deckt somit den im Projekt festgelegten Zielzeitraum von \textbf{maximal 3–4 Jahren} typischer Gerätelebensdauer ab (siehe Anhang~\ref{anhang:meeting16102025}).  
Diese Konfiguration gewährleistet, dass die Anwendung moderne Android-Technologien nutzt, ohne dabei die Kompatibilität zu verbreitet eingesetzten Geräten zu verlieren.  
Sie entspricht zudem den Empfehlungen der offiziellen Android-Dokumentation \parencite{noauthor_sdk_nodate}.

\subsubsection{Wesentliche Bibliotheken und Abhängigkeiten}
Im Folgenden sind die wichtigsten externen Bibliotheken aufgeführt, die in der \texttt{build.gradle}-Datei des App-Moduls verwendet werden:

\begin{itemize}
    \item \textbf{Jetpack Compose} (\texttt{androidx.compose.ui, material, material3}):  
    Für deklaratives UI-Design mit reaktiver State-Verwaltung.
    
    \item \textbf{AndroidX Core und Lifecycle} (\texttt{core-ktx, lifecycle-runtime-ktx, activity-compose}):  
    Basisfunktionen und Lifecycle-Unterstützung für moderne Android-Komponenten.
    
    \item \textbf{Navigation Compose} (\texttt{androidx.navigation:navigation-compose}):  
    Zur Verwaltung von Navigation und Routen in Compose-Anwendungen.
    
    \item \textbf{Hilt (Dependency Injection)} (\texttt{com.google.dagger:hilt-android}, \texttt{androidx.hilt:hilt-navigation-compose}):  
    Ermöglicht modulare, testbare und skalierbare Architektur durch Dependency Injection.
    
    \item \textbf{DataStore} (\texttt{androidx.datastore:datastore-preferences}):  
    Moderne, asynchrone und typsichere Alternative zu SharedPreferences.
    
    \item \textbf{JUnit 5 und Android Test Libraries}:  
    Einheitliches Testframework mit Unterstützung für Unit- und Instrumentation-Tests (\texttt{org.junit.jupiter}, \texttt{androidx.test.ext:junit}, \texttt{espresso-core}).
    
    \item \textbf{Accompanist} (\texttt{pager, pager-indicators, webview, permissions}):  
    Erweiterungsbibliotheken von Google zur Ergänzung von Jetpack Compose (z.~B. ViewPager, WebView, Berechtigungsdialoge).
    
    \item \textbf{ZXing \& AltBeacon} (\texttt{com.google.zxing:core}, \texttt{org.altbeacon:android-beacon-library}):  
    QR-Code-Scanning und Beacon-Erkennung für die \texttt{Trail}-Funktion.
    
    \item \textbf{Gson}:  
    Für JSON-Serialisierung und -Deserialisierung.
\end{itemize}