Im Rahmen des Wirtschaftsprojekts \emph{HSLU Mobile Apps – Android Jetpack Compose – AI First}
wurde eine bestehende Android-Applikation der Hochschule Luzern technisch neu umgesetzt.
Ausgangslage war eine veraltete Android-App auf Basis von XML-Layouts sowie eine moderne
iOS-Applikation, welche als Referenz diente. Ziel des Projekts war es, eine zeitgemässe
Android-Applikation mit Jetpack Compose zu realisieren, die funktional an die iOS-Version
angeglichen ist, aktuelle Android-Technologien nutzt und produktiv im Google Play Store
veröffentlicht werden kann.

Die Umsetzung erfolgte über ein agiles Vorgehen mit zweiwöchigen Sprints. Zu Projektbeginn
wurde gemeinsam mit dem Auftraggeber ein klarer Rahmen definiert, inklusive regelmässiger
Status-Meetings und transparenter Aufgabenverwaltung über GitLab. Die Entwicklung wurde
modular aufgebaut. Gemeinsame,
fachlich unabhängige Komponenten wurden in zentralen \texttt{common}-Modulen umgesetzt,
während einzelne Features wie News, Blog, Mensa, Events oder Raumsuche jeweils als eigene
Module realisiert wurden. Diese Struktur wurde dabei von der bestehenden iOS-Applikation
übernommen, um eine einheitliche Architektur über beide Plattformen hinweg sicherzustellen.

Ein Schwerpunkt des Projekts lag auf dem AI-First-Ansatz.
KI-gestützte Entwicklungswerkzeuge wurden getestet und dann aktiv in den Entwicklungsprozess eingebunden. 
AI wurde unter anderem für Code-Analysen,
Übersetzungen von bestehendem iOS-Code nach Kotlin, Unterstützung bei neuen Features
sowie für das Testing eingesetzt. Die Erfahrungen zeigen, dass AI insbesondere
bei klar abgegrenzten Aufgaben und gut formulierten Prompts eine deutliche Zeitersparnis
ermöglicht, gleichzeitig aber weiterhin fundierte Programmierkenntnisse notwendig bleiben.

Als Resultat steht eine moderne, modular aufgebaute Android-Applikation auf Basis von
Jetpack Compose zur Verfügung, die Multi-Tenant-fähig ist, aktuelle Android-SDKs
unterstützt und automatisiert über CI/CD und Fastlane ausgeliefert werden kann.
Ergänzend dazu wurden alle Projektartefakte wie Dokumentation, Risikoanalyse,
Sprint-Backlogs, Statusberichte und Protokolle vollständig erstellt, sodass eine
Weiterentwicklung des Projekts problemlos möglich ist.