In der aktuellen Ausgangslage existieren drei unterschiedliche mobile Applikationen, die sich sowohl hinsichtlich ihres technologischen Stands als auch ihres Funktionsumfangs 
deutlich voneinander unterscheiden. Diese Situation erfordert eine klare Einordnung der bestehenden Lösungen sowie eine strategische Ausrichtung für die Weiterentwicklung der 
Android-Plattform.

Zum einen gibt es eine veröffentlichte Android-Applikation, die auf klassischen XML-Layouts basiert. Diese Anwendung ist bereits seit längerer Zeit im Einsatz, nutzt jedoch weder 
alle verfügbaren Funktionen noch moderne Android-Entwicklungskonzepte. Wichtige aktuelle Technologien und Best Practices, wie sie heute im Android-Ökosystem etabliert sind, wurden 
in dieser App nicht oder nur unzureichend berücksichtigt. Dadurch entspricht sie nicht mehr dem aktuellen Stand der Technik und weist sowohl in Bezug auf 
Wartbarkeit als auch auf Benutzererlebnis klare Defizite auf. Erweiterungen oder Anpassungen sind dadurch aufwendiger und weniger nachhaltig umsetzbar.

Parallel dazu existiert eine iOS-Applikation, die sich auf einem aktuellen technischen Stand befindet und ebenfalls bereits veröffentlicht ist. Diese App nutzt moderne 
Entwicklungsansätze und implementiert die gewünschten Funktionen vollständig und zeitgemäss. Sie stellt damit eine stabile, ausgereifte und zukunftsfähige Lösung dar. 
Aufgrund ihres aktuellen Zustands und der vollständigen Feature-Abdeckung dient diese iOS-Applikation als fachliche und funktionale Referenz. An ihr kann und soll sich 
die Weiterentwicklung der Android-Anwendungen orientieren, insbesondere im Hinblick auf Funktionsumfang, Benutzerführung und technisches Design.

Als dritte Variante liegt zudem eine Android-Applikation auf Basis von Jetpack Compose vor. Diese App ist aktuell nicht veröffentlicht und ebenfalls veraltet. 
Obwohl Jetpack Compose grundsätzlich ein modernes UI-Framework darstellt, wurde die Anwendung seit längerer Zeit nicht mehr aktualisiert und nutzt weder die aktuellen 
Möglichkeiten von Compose noch moderne Architektur- und State-Management-Konzepte. Funktional und technisch kann sie daher nicht als zeitgemäss betrachtet werden. 
Gleichzeitig bietet sie jedoch eine gute Grundlage für eine moderne Android-Neuentwicklung, da sie bereits auf einem deklarativen UI-Ansatz basiert.

Ziel ist es nun, diese Android-Jetpack-Compose-Applikation gezielt zu modernisieren und weiterzuentwickeln. Dabei soll sie technologisch auf den neuesten Stand 
gebracht und funktional an die bestehende iOS-Applikation angeglichen werden. Die iOS-App dient hierbei als Referenz für Features, Benutzererlebnis und funktionale Vollständigkeit. 
Durch diese Ausrichtung kann sichergestellt werden, dass künftig eine konsistente und moderne App-Landschaft auf beiden Plattformen existiert, die sowohl den aktuellen technischen 
Standards entspricht als auch eine einheitliche Nutzererfahrung bietet.
