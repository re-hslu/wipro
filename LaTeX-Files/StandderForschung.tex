\section{Apps}\vspace{-0.25em}
Die Hochschule Luzern betreibt für die Departemente Informatik sowie Technik \& Architektur
eigene mobile Applikationen auf den Plattformen iOS und Android. In den vergangenen
Wirtschaftsprojekten und Forschungsarbeiten wurden diese Apps laufend weiterentwickelt.
Dabei arbeiteten jeweils unterschiedliche Studierende an den Projekten, jedoch nicht immer
parallel auf beiden Plattformen. Aus diesem Vorgehen sind mehrere Projekte entstanden, die sich heute in unterschiedlichen
Entwicklungsständen befinden. Alle Projekte werden zentral auf GitLab verwaltet und
versioniert.

Aktuell existieren drei unterschiedliche Versionen der Applikation:
\begin{itemize}
    \item \texttt{ios-swiftui-multitenant}
    \item \texttt{android-xml-multitenant}
    \item \texttt{android-jetpackcompose-multitenant}
\end{itemize}

\vspace{0.3em}
\begin{figure}[H]
    \centering
    \includegraphics[width=0.55\textwidth]{Fotos/projekte-gitlab.png}
    \caption{Projekte in GitLab}
    \label{fig:projekte-gitlab}
\end{figure}

Die Projekte \texttt{ios-swiftui-multitenant} sowie \texttt{android-xml-multitenant} verfügen
jeweils über eine produktive Version der Applikation, welche im Apple App Store respektive
im Google Play Store veröffentlicht ist. (Diese produktiven Stände sind in der Abbildung 2
grün markiert.)

Zusätzlich existiert im Projekt \texttt{ios-swiftui-multitenant} ein Entwicklungsstand
(\textit{Dev-Branch}), der bereits weitere Features enthält, jedoch noch nicht veröffentlicht
wurde. Dieser Entwicklungsstand stellt aktuell den funktional vollständigsten und
modernsten Stand der Applikation dar. (Er ist in der Abbildung 2 blau dargestellt und dient
als Referenz für die Android-Neuentwicklung.)

Das Projekt \texttt{android-jetpackcompose-multitenant} ist aktuell nicht produktiv.
An diesem Projekt wurde seit längerer Zeit nicht mehr aktiv gearbeitet, weshalb es
technisch als veraltet betrachtet werden muss. Ziel des Projekts ist es, den aktuellen Entwicklungsstand aus dem
\texttt{ios-swiftui-multitenant}-Dev-Branch funktional auf Android mit
Jetpack Compose nachzubauen und weiterzuentwickeln.

\vspace{0.3em}
\begin{figure}[H]
    \centering
    \includegraphics[width=0.55\textwidth]{Fotos/projekte.png}
    \caption{Projekte mit (relevanten) Branches}
    \label{fig:branches}
\end{figure}

\vspace{0.5cm}

Die folgenden Abbildungen zeigen den Stand der bestehenden mobilen Applikationen
im September 2025, also zum Zeitpunkt vor Projektbeginn.

\begin{figure}[H]
    \centering
    \begin{subfigure}[b]{0.23\textwidth}
        \centering
        \includegraphics[width=\textwidth]{Fotos/android-xml-multitenant.png}
        \caption{android-xml-multitenant (main)}
        \label{fig:android-xml-multitenant}
    \end{subfigure}
    \hspace{0.3cm}
    \begin{subfigure}[b]{0.23\textwidth}
        \centering
        \includegraphics[width=\textwidth]{Fotos/ios-swiftui-prod.png}
        \caption{ios-swiftui-multitenant (main)}
        \label{fig:ios-swiftui-prod.png}
    \end{subfigure}
    \hspace{0.3cm}
    \begin{subfigure}[b]{0.23\textwidth}
        \centering
        \includegraphics[width=\textwidth]{Fotos/ios-swiftui-dev.png}
        \caption{ios-swiftui-multitenant (dev)}
        \label{fig:ios-swiftui-dev.png}
    \end{subfigure}
    \caption{Screenshots der bestehenden Apps}
    \label{fig:bestehendeappsstand}
\end{figure}

\section{Backend}

Für alle Applikationen existiert ein gemeinsames Backend, auf das sowohl die iOS- als auch
die Android-Apps zugreifen. Über dieses Backend werden unter anderem Informationen zu den
einzelnen Modulen geladen. 

Dieses Konzept ermöglicht eine Entkopplung zwischen App und Inhalt. Änderungen an
Daten oder Modulinhalten können zentral im Backend vorgenommen werden, ohne dass die
Applikationen neu veröffentlicht werden müssen. Dadurch wird die Wartbarkeit verbessert
und neue Inhalte können schneller und flexibler bereitgestellt werden (siehe ~\ref{anhang:backend}).