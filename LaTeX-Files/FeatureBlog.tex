\subsection{Blog}

Das Feature \textit{FeatureBlog} baut konzeptionell auf derselben Architektur wie das zuvor 
beschriebene \textit{FeatureNews} auf, verfolgt jedoch einen leicht anderen Zweck: 
Es stellt die Blog-Beiträge der HSLU aus der WordPress-Installation \url{https://hub.hslu.ch/informatik/} dar. 
Der grundlegende technische Aufbau mit \texttt{ModuleLoader}, \texttt{DataLoader}, 
JSON-Synchronisation sowie HTML-Parsing ist identisch zum News-Feature, 
wodurch sich viele Komponenten wiederverwenden liessen. Auch hier ist ein Umschalten zwischen WebView- 
und JSON-View im Backend möglich, sodass bei Bedarf  zwischen beiden Darstellungsvarianten gewechselt werden kann.


Die Umsetzung der UI folgt dabei denselben Prinzipien wie bei den News: Karten-basierte Übersicht, Detailansicht 
als Overlay/Dialog und Rendering des Inhalts über den bestehenden Parser und den \texttt{BlogContentRenderer}.

\vspace{0.5cm}

\begin{figure}[H]
    \centering
    \begin{subfigure}[b]{0.23\textwidth}
        \centering
        \includegraphics[width=\textwidth]{Fotos/feature-screenshots/blog_json.png}
        \caption{Übersicht}
        \label{fig:blog_json}
    \end{subfigure}
    \hspace{0.3cm}
    \begin{subfigure}[b]{0.23\textwidth}
        \centering
        \includegraphics[width=\textwidth]{Fotos/feature-screenshots/blog_json_detail.png}
        \caption{Detailansicht}
        \label{fig:blog_json_detail}
    \end{subfigure}
    \caption{Screenshots des Blog-Features}
    \label{fig:blog_json_feature}
\end{figure}

\vspace{1.5cm}

\begin{figure}[H]
    \centering
    \includegraphics[width=0.65\textwidth]{Fotos/feature-screenshots/backend-json-web.png}
    \caption{Screenshot aus dem Backend: Umschalten der Views (Blog \& News)}
    \label{fig:backend-json-web}
\end{figure}