% ================== Kapitel: Methoden ============================================

\section{Gewünschte Methoden und Vorgehen}
Inkrementelles, iteratives, agiles Vorgehen.  
Regelmässige Treffen mit dem Auftraggeber finden statt, inklusive Statusbericht am Vorabend.  
Der Statusbericht umfasst erledigte Aufgaben, den aktuellen Zwischenstand, eine Risikobewertung sowie die nächsten geplanten Arbeiten.  
Zur Nachvollziehbarkeit von Fortschritt und Aufgaben wird ein Projektmanagement-Tool verwendet, das eine transparente Verfolgung der Issues ermöglicht.

\section{Kreativität, Methoden, Innovation}
Im Projekt sollen moderne Technologien eingesetzt und evaluiert werden, wie beispielsweise AI-gestützte Entwicklungswerkzeuge, Jetpack Compose oder CI/CD-Automatisierungen.  
Die Unterstützung durch künstliche Intelligenz wird aktiv in den Entwicklungsprozess eingebunden, dokumentiert und reflektiert.  
Je nach Erkenntnissen im Projektverlauf wird die Gewichtung des AI-Anteils flexibel angepasst, um die bestmöglichen Resultate zu erzielen.

\section{Projektmanagement}

\subsection{Stakeholder}
\begin{itemize}
    \item Auftraggeber / Dozent: Jürg Nietlispach
    \item Experte: Martin Vogel
    \item Projektteam: Raphael Eiholzer, Samuel Kurmann
\end{itemize}

\subsection{Agiles Vorgehen}
Im Projekt wurde nach dem agilen Vorgehensmodell \textit{Scrum} gearbeitet.  
Die Sprints wurden jeweils über zwei Wochen definiert, wodurch sich insgesamt sieben Sprints ergaben.  
Am Ende jedes Sprints fand eine Besprechung mit dem Auftraggeber statt, um den Fortschritt zu präsentieren und den nächsten Sprint zu planen.  

Zu Beginn wurden alle Aufgaben in GitLab als Issues erfasst und in das Product Backlog aufgenommen.  
Während der Sprintplanung wurden die Issues nach Priorität ausgewählt und dem kommenden Sprint zugeordnet.  
Dieses Vorgehen ermöglichte eine strukturierte und transparente Projektabwicklung mit klaren Zuständigkeiten und Fortschrittskontrolle.  

\begin{figure}[H]
    \centering
    \includegraphics[width=0.4\textwidth]{Fotos/sprint_cycle.png}
    \caption{Darstellung des Scrum-Prozesses im Projekt} \parencite{atlassian_was_nodate}
    \label{fig:scrum}
\end{figure}

\subsection{Planung}
Zu Beginn des Projekts wurde eine detaillierte Planung der auszuführenden Arbeitsschritte erstellt.  
Dazu erfolgte zunächst eine Analyse der bestehenden iOS-App und ein Vergleich mit der bisherigen Jetpack-Compose-Version.  
Die wichtigsten Meilensteine wurden in einem Plan zusammengeführt, um den Projektfortschritt wöchentlich zu überprüfen und Abweichungen frühzeitig zu erkennen.  

\begin{figure}[H]
    \centering
    \includegraphics[width=0.9\textwidth]{Fotos/meilensteinplan.png}
    \caption{Meilensteinplan des Projekts}
    \label{fig:meilensteinplan}
\end{figure}

Die definierten Meilensteine wurden in GitLab abgebildet und die einzelnen Arbeiten als Issues erfasst und den entsprechenden Meilensteinen zugeordnet.  
Die Aufgaben wurden in zwei Kategorien unterteilt:
\begin{itemize}
    \item \textit{Common-Arbeiten}: allgemeine App-Funktionalitäten, die von mehreren Modulen genutzt werden (z. B. Netzwerkkommunikation)
    \item \textit{Feature-Arbeiten}: spezifische App-Funktionen oder Menüpunkte, die unabhängig voneinander entwickelt werden können
\end{itemize}

Die Aufgaben wurden so gestaltet, dass sie von einer einzelnen Person vollständig bearbeitet werden konnten.  
Die Reihenfolge der Issues wurde zu Projektbeginn grob festgelegt, aber jeweils in der Sprintplanung überprüft und bei Bedarf angepasst.  

\begin{figure}[H]
    \centering
    \includegraphics[width=0.9\textwidth]{Fotos/sprintplanung-1.png}
    \caption{Beispielhafte Darstellung des Backlogs in GitLab}
    \label{fig:backlog}
\end{figure}

Jedes Issue wurde als User Story formuliert und mit einer Definition of Done (DoD) versehen.  
Die Projektmitglieder konnten sich selbst Issues zuweisen, wodurch eine klare Arbeitsaufteilung gewährleistet war.  
Für jede Aufgabe wurde die aufgewendete Zeit direkt im Issue dokumentiert, um den Arbeitsaufwand über die gesamte Projektlaufzeit nachvollziehen zu können.

\begin{figure}[H]
    \centering
    \includegraphics[width=0.9\textwidth]{Fotos/issue-userstory-dod.png}
    \caption{Beispiel einer User Story mit Definition of Done (DoD)}
    \label{fig:userstory}
\end{figure}

\subsection{Risikoanalyse}
Zu Beginn des Projekts wurde eine Risikoanalyse erstellt, um die grössten Risiken frühzeitig zu identifizieren und entsprechende Gegenmassnahmen zu planen.  
Die Risiken wurden hinsichtlich Eintrittswahrscheinlichkeit und Auswirkung bewertet.  
Das Produkt dieser beiden Faktoren bestimmte den Gesamtrisiko-Wert und erlaubte es, die kritischsten Risiken zu priorisieren.  

Für jedes Risiko wurden präventive Massnahmen festgelegt, um die Eintrittswahrscheinlichkeit oder die Auswirkungen zu minimieren.  
Nach Umsetzung dieser Massnahmen wurde die Bewertung erneut vorgenommen, um verbleibende Risiken zu identifizieren und ihre Entwicklung über die Projektlaufzeit zu beobachten.  

\begin{figure}[H]
    \centering
    \includegraphics[width=0.4\textwidth]{Fotos/risikomatrix.png}
    \caption{Risikoanalyse und Risikomatrix des Projekts}
    \label{fig:risikomatrix}
\end{figure}

Die Risikoanalyse wurde während der gesamten Projektdauer regelmässig überprüft und aktualisiert.  
Am Ende jedes Sprints diskutierten die Studierenden gemeinsam die aktuelle Risikosituation und leiteten mögliche Anpassungen ein.  
Die drei grössten Risiken wurden zusätzlich an den Auftraggeber übermittelt, um Transparenz über den Projektfortschritt und potenzielle Herausforderungen sicherzustellen.