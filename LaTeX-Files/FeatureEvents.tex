\subsection{Events}

Das Feature \textit{FeatureEvents} erweitert die App um die Möglichkeit, 
aktuelle Veranstaltungen der Hochschule Luzern direkt anzuzeigen. 
Im Gegensatz zu den zuvor beschriebenen Features \textit{News} und \textit{Blog}, 
die ihre Inhalte über die WordPress-API beziehen, basiert dieses Feature auf 
der Sitecore-Plattform der HSLU. Sitecore stellt Inhalte strukturiert über 
eine JSON-basierte API bereit und ist damit eine gute Grundlage für eine 
native Darstellung in der App. \parencite{sitecore_sitecore_nodate}

Die grundlegende Architektur folgt weiterhin dem bekannten Muster: 
Ein \texttt{ModuleLoader} stellt die Moduldefinitionen bereit, 
während der \texttt{EventsDataLoader} die Eventdaten lädt, in 
Kotlin-Domainobjekte umwandelt und diese der UI zur Verfügung stellt. 
Das JSON-Processing unterscheidet sich jedoch in einigen Punkten von News und Blog:
\begin{itemize}
\item Die Sitecore-API liefert andere Feldstrukturen, z.,B. separate Eigenschaften für Start- und Enddatum.
\item Der Eventtext enthält oft komplexere HTML-Fragmente, weshalb das Parsing stärker variieren kann.
\item Die Events besitzen zusätzliche Metadaten wie \textit{organizer}, \textit{event type} oder Kategorien.
\end{itemize}

Die Daten werden in der App wieder in zwei Compose-Views dargestellt: 
einer Übersicht aller Events sowie einer Detailansicht (\texttt{EventsDetailSitecoreView}),
 die die Inhalte strukturiert rendert und zusätzlich den Link zur Eventseite der HSLU anbietet.

Um sicherzustellen, dass nur relevante Events des Departements Informatik angezeigt werden, 
kann das Backend Sitecore-spezifische Filter konfigurieren. So kann etwa der 
Query-Parameter \texttt{filters[]=1621} gesetzt werden, ohne dass dieser Wert im App-Code hinterlegt sein muss. 
Die Filterlogik bleibt damit vollständig backend-gesteuert.

\vspace{0.5cm}

\begin{figure}[H]
    \centering
    \begin{subfigure}[b]{0.23\textwidth}
        \centering
        \includegraphics[width=\textwidth]{Fotos/feature-screenshots/event_json.png}
        \caption{Übersicht Events}
        \label{fig:events_json}
    \end{subfigure}
    \hspace{0.3cm}
    \begin{subfigure}[b]{0.23\textwidth}
        \centering
        \includegraphics[width=\textwidth]{Fotos/feature-screenshots/event_json_detail.png}
        \caption{Detail-Ansicht}
        \label{fig:events_json_detail}
    \end{subfigure}
    \caption{Screenshots des Events-Features}
    \label{fig:events_feature}
\end{figure}
