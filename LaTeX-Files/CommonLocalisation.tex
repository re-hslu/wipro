\subsection {Lokalisierung}

Die Lokalisierung stellt die sprachabhängige Darstellung von Texten und Labels in der App sicher. 
Es basiert auf dem bestehenden Android-Resource-System und wurde so erweitert, 
dass dynamisch geladene Inhalte (z.\,B. Menüpunkte aus dem Bootstrapping) ebenfalls lokalisiert dargestellt werden können.

\subsubsection*{Neue Umsetzung}
Die grundlegende Logik blieb unverändert, da die Mehrsprachigkeit bereits funktionierte. 
Folgende Anpassungen wurden vorgenommen:
\begin{itemize}
    \item Sprachwahl erfolgt weiterhin automatisch über die Geräteeinstellung, keine separate App-Sprache erforderlich.
    \item Ressourcenbasiertes System: Deutsch als Standard in \texttt{values/}, Englisch in \texttt{values-en/} 
    (appweit und in jedem Feature-Modul).
    \item Nutzung der UI-Strings erfolgt wie üblich über \texttt{stringResource(R.string...)} 
    bzw. \texttt{context.getString(...)}.
\end{itemize}

\subsubsection*{Übersetzung dynamischer Labels}
Für dynamisch geladene Inhalte, etwa Menüpunkte aus Remote-DTOs, 
wird die Sprache zur Laufzeit anhand der Systemeinstellung bestimmt. 
Die Zuordnung erfolgt über \texttt{Locale.getDefault()}:
\begin{itemize}
    \item Deutsch (\texttt{de}) → LabelDE / BootstrapResourceUrlDE
    \item Englisch (Standard) → LabelEN / BootstrapResourceUrlEN
\end{itemize}

Ein Beispiel der Umsetzung in Kotlin:
\begin{lstlisting}
val Label: String
    get() = when (Locale.getDefault().language.lowercase(Locale.getDefault())) {
        "de" -> LabelDE ?: LabelEN ?: "Unbenannt"
        else -> LabelEN ?: LabelDE ?: "Unnamed"
    }
\end{lstlisting}

\subsubsection*{Beispiel: Ressourcenbasiertes UI-Label}
\begin{lstlisting}[language=XML, caption={Default (Deutsch) vs. Englisch in strings.xml}]
<!-- values/strings.xml -->
<string name="Roomsearch_NavItem">Raumsuche</string>
<string name="Settings_NavItem">Einstellungen</string>

<!-- values-en/strings.xml -->
<string name="Roomsearch_NavItem">Roomsearch</string>
<string name="Settings_NavItem">Settings</string>
\end{lstlisting}

Damit ist die App vollständig zweisprachig ausgelegt. 
Labels aus Remote-Datenquellen werden zur Laufzeit automatisch auf die passende Sprache gemappt, 
wodurch eine konsistente Benutzererfahrung entsteht.