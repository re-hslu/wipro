\subsection{CampusRoomSearch}

Das CampusRoomSearch Feature ermöglicht es Benutzern, Räume auf dem Campus zu suchen und zu finden. 
Es stellt interaktive PDF-Karten bereit, die Campus-Übersichten und Gebäudepläne enthalten, und ermöglicht es Benutzern, 
durch Klicken auf annotierte Bereiche in den PDFs zu navigieren und detaillierte Informationen zu Räumen, Gebäuden und Etagen abzurufen. 
Das Feature unterstützt sowohl deutsche als auch englische Sprachversionen und speichert PDF-Dateien lokal für Offline-Zugriff.

\vspace{0.5cm}

\begin{figure}[H]
    \centering
    \includegraphics[width=0.9\textwidth]{Fotos/feature-screenshots/campusroomsearch_diagramm.png}
    \caption{Ablaufdiagramm}
    \label{fig:campusroomsearch_diagramm}
\end{figure}

\subsubsection*{Ziel und Motivation}

Das Hauptziel des CampusRoomSearch Features besteht darin, Studierenden und Mitarbeitenden eine intuitive Möglichkeit zu bieten, 
Räume auf dem Campus zu finden. Durch die Verwendung von interaktiven PDF-Karten mit Annotationen können Benutzer direkt auf Bereiche in den Karten klicken, 
um zu detaillierten Gebäude- und Etagenplänen zu navigieren. Dies bietet eine benutzerfreundliche Alternative zu textbasierten Suchfunktionen.

Ein weiteres wichtiges Ziel ist die Offline-Funktionalität. Da Campus-Pläne und Gebäudekarten häufig benötigt werden, 
werden alle PDF-Dateien lokal gespeichert, sodass das Feature auch ohne aktive Netzwerkverbindung vollständig funktionsfähig ist. 
Die Synchronisation erfolgt intelligent basierend auf Timestamp-Vergleichen, um unnötige Downloads zu vermeiden.

\subsubsection*{Umsetzung / Funktionsweise}

Die Implementierung basiert auf einer zweistufigen Architektur: \texttt{CampusRoomSearchModuleLoader} lädt die Liste der verfügbaren Campus-Standorte, 
während \texttt{CampusRoomSearchDataLoader} die detaillierten Daten für einen spezifischen Campus-Standort lädt, 
einschliesslich aller PDF-Dateien.

Der \texttt{CampusRoomSearchModuleLoader} erbt von \texttt{CommonApplicationBaseModuleLoader} und erweitert die Standard-Funktionalität, 
um nach dem Laden der Campus-Liste automatisch die detaillierten Daten für alle Items zu laden.

Die \texttt{loadRemoteDetailedData()}-Methode lädt die detaillierten Campus-Daten von der API und lädt anschliessend alle zugehörigen PDF-Dateien herunter.

Die \texttt{PDFRoomSearchView} Komponente rendert PDF-Dateien mit Android's \texttt{PdfRenderer} und unterstützt interaktive Annotationen. Die Komponente lädt PDFs aus der Room-Datenbank und ermöglicht Zoom- und Pan-Gesten:

\begin{lstlisting}[language=Java]
@Composable
fun PDFRoomSearchView(
    pdfData: ByteArray,
    annotations: List<Annotation>,
    onAnnotationClick: (Annotation) -> Unit
) {
    // PDF-Rendering mit PdfRenderer
}
\end{lstlisting}

Die Annotation-Erkennung funktioniert durch Koordinaten-Transformation. Wenn ein Benutzer auf das PDF klickt, 
werden die Tap-Koordinaten basierend auf dem aktuellen Zoom- und Pan-Zustand transformiert, um die entsprechende Annotation zu finden:

\begin{lstlisting}[language=Java]
fun transformTapCoordinates(
    tapX: Float, tapY: Float,
    zoom: Float, panX: Float, panY: Float
): Pair<Float, Float> {
    return ((tapX - panX) / zoom, (tapY - panY) / zoom)
}
\end{lstlisting}

Der \texttt{RoomsearchViewModel} verwaltet die Suche nach Räumen und die Interaktion mit den SVG-Karten. 
Er verwendet JavaScript-Interfaces, um Kommunikation zwischen WebView und Android-Code zu ermöglichen:

\begin{lstlisting}[language=Java]
interface IJavascriptHandler {
    @JavascriptInterface
    fun onBuildingClick(buildingId: String)
}
\end{lstlisting}

\subsubsection*{Weitere Informationen}

\begin{table}[H]
\centering
\begin{tabularx}{\textwidth}{|l|X|}
\hline
\textbf{Funktionalität} & \textbf{Beschreibung} \\
\hline
\textbf{Hybride Speicherstrategie} & 
Das Feature verwendet eine hybride Speicherstrategie: JSON-Daten werden in der Room-Datenbank gespeichert, während PDF-Dateien als \texttt{ByteArray} in der \texttt{FileEntity} Tabelle gespeichert werden. 
Dies ermöglicht eine effiziente Speicherung und schnellen Zugriff auf die Dateien, ohne dass externe Dateisystem-Zugriffe erforderlich sind. \\
\hline
\textbf{PDF-Annotationen} & 
Die PDF-Annotationen werden aus den PDF-Metadaten extrahiert und als unsichtbare, aber klickbare Bereiche über dem PDF gerendert. 
Dies ermöglicht eine präzise Interaktion, während das ursprüngliche PDF-Layout erhalten bleibt. Die Koordinaten-Transformation berücksichtigt Zoom- und Pan-Zustände für eine genaue Erkennung. \\
\hline
\textbf{Sprachauswahl} & 
Die Sprachauswahl erfolgt automatisch basierend auf der System-Locale. Falls die bevorzugte Sprache nicht verfügbar ist, wird automatisch auf die alternative Sprache zurückgegriffen. 
Dies gewährleistet, dass das Feature immer funktionsfähig ist, unabhängig von der verfügbaren Sprachversion. \\
\hline
\textbf{Intelligente Synchronisation} & 
Die Synchronisation verwendet Timestamp-Vergleiche, ähnlich wie andere Module. Der \texttt{CampusRoomSearchDataLoader} prüft zuerst, ob lokale Daten vorhanden sind, und lädt nur dann Remote-Daten, wenn eine Aktualisierung notwendig ist oder keine lokalen Daten vorhanden sind. 
Dies reduziert unnötige Downloads und spart Bandbreite. \\
\hline
\textbf{Hierarchische Navigation} & 
Das Feature unterstützt mehrere Campus-Standorte, wobei jeder Standort seine eigenen Gebäude, Etagen und Räume hat. 
Die Navigation erfolgt hierarchisch: Campus → Gebäude → Etage → Raum, wobei jede Ebene durch entsprechende PDF-Karten und Annotationen dargestellt wird. \\
\hline
\textbf{WebView-Integration} & 
Die WebView-Integration für SVG-Karten ermöglicht interaktive Gebäudekarten, bei denen Benutzer auf Gebäude klicken können, um direkt zu den entsprechenden Etagenplänen zu navigieren. 
Die JavaScript-Interface-Implementierung ermöglicht eine bidirektionale Kommunikation zwischen der WebView und dem Android-Code. \\
\hline
\textbf{Offline-Funktionalität} & 
Alle PDF-Dateien werden lokal gespeichert, sodass das Feature auch ohne aktive Netzwerkverbindung vollständig funktionsfähig ist. 
Die Synchronisation erfolgt intelligent basierend auf Timestamp-Vergleichen, um unnötige Downloads zu vermeiden. \\
\hline
\textbf{PDF-Rendering} & 
Die PDF-Dateien werden mit Android's \texttt{PdfRenderer} gerendert und unterstützen Zoom- und Pan-Gesten. 
Die Komponente lädt PDFs aus der Room-Datenbank und ermöglicht eine flüssige Interaktion mit den Karten. \\
\hline
\end{tabularx}
\caption{Wichtigste Funktionalitäten des \texttt{CampusRoomSearch} Features}
\label{tab:campusroomsearch-features}
\end{table}
