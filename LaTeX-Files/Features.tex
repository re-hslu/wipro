Die Applikation ist konsequent feature-orientiert aufgebaut. 
Sämtliche fachlichen Funktionalitäten der App sind in eigenständigen Feature-Modulen gekapselt, 
welche sich im Verzeichnis \texttt{/features} befinden.
Ein Feature repräsentiert dabei jeweils einen abgegrenzten Anwendungsbereich (und Menüpunkt) der App,
wie beispielsweise \textit{Blog}, \textit{News}, \textit{Mensa}, \textit{Events}, \textit{Raumsuche},
\textit{Stundenplan}, \textit{Einstellungen} oder \textit{AppInit}.
Jedes Feature ist als separates Android-Library-Modul umgesetzt und kann unabhängig entwickelt,
getestet und gewartet werden. Im Falle unserer Android-App hat auch jedes Feature einen 
eigenen API-Endpunnkt.

\begin{figure}[H]
    \centering
    \includegraphics[width=0.4\textwidth]{Fotos/features_projekt.png}
    \caption{Features in Android-Studio}
    \label{fig:features_projekt}
\end{figure}

\subsubsection*{Ziel und Rolle der Features}

Features bilden die funktionale Ebene der Applikation.
Sie enthalten alle Bestandteile, die für eine bestimmte Funktionalität erforderlich sind.
Durch diese Aufteilung wird sichergestellt, dass einzelne Funktionen klar voneinander getrennt sind
und Änderungen möglichst lokal auf ein einzelnes Feature beschränkt bleiben.

Ein zentrales Architekturprinzip ist, dass Features ausschliesslich von den
\textit{Common}-Modulen abhängen dürfen, jedoch nicht voneinander (Einzige Ausnahme: \textit{AppInit}).
Dadurch werden zyklische Abhängigkeiten vermieden und die Architektur bleibt langfristig erweiterbar
und wartbar.

\begin{figure}[H]
    \centering
    \includegraphics[width=0.16\textwidth]{Fotos/abhaengigkeiten.png}
    \caption{Abhängigkeiten in Pfeilrichtung erlaubt/vorhanden}
    \label{fig:abhaengigkeiten}
\end{figure}

\subsubsection*{Aufbau eines Features}

Alle Features folgen grundsätzlich der gleichen Struktur.
Die Struktur und Benennung der Ordner wurden bewusst an der iOS-App orientiert,
sodass Entwickler, die an beiden Projekten arbeiten,
die entsprechenden Code-Stellen plattformübergreifend schnell wiederfinden.

Ein Feature gliedert sich typischerweise in folgende Bereiche:

\begin{itemize}
    \item \textbf{Domain:} 
    Enthält die fachlichen Modelle und Domain-Objekte des Features.
    Diese Schicht ist unabhängig von Android- oder UI-spezifischen Frameworks
    und bildet die Grundlage für Business-Logik und Datenverarbeitung.

    \item \textbf{Services:} 
    Beinhaltet die Logik zum Laden, Synchronisieren und Verarbeiten von Daten.
    Hier befinden sich unter anderem Loader-Klassen, welche für den Zugriff auf Backend-APIs,
    die lokale Speicherung sowie Synchronisationsmechanismen verantwortlich sind.

    \item \textbf{View:} 
    Enthält die UI-Komponenten des Features, umgesetzt mit Jetpack Compose.
    Die Views reagieren reaktiv auf Zustandsänderungen der ViewModels
    und stellen die Daten dem Nutzer dar.

    \item \textbf{Resources:} 
    Feature-spezifische Ressourcen wie Strings oder weitere Assets,
    die unabhängig von anderen Features gepflegt werden können.
\end{itemize}

\begin{figure}[H]
    \centering
    \includegraphics[width=0.4\textwidth]{Fotos/features_projekt_event.png}
    \caption{Struktur der Features am Beispiel \textit{Event}}
    \label{fig:features_projekt_event}
\end{figure}

Das Feature \textit{AppInit} unterscheidet sich dabei grundsätzlich von den übrigen Features,
da es nicht eine fachliche Funktionalität bereitstellt,
sondern für die initiale Konfiguration, das Bootstrapping und die Steuerung
der Applikation verantwortlich ist.