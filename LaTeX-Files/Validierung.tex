\section{Validierung}

\subsection{Technische Validierung}

\subsubsection{Bisherige Testabdeckung}
Die bisherigen App-Projekte der Hochschule
(\textit{android-xml-multitenant} und \textit{ios-swiftui-multitenant})
verfügten bislang über kaum automatisierte Testabdeckung.
Da wir diesen Aspekt jedoch als wichtig erachten, haben wir uns zum Ziel
gesetzt, zentrale Funktionen mit ausführbaren Tests abzusichern,
um die Funktionalität und Stabilität der App besser gewährleisten zu können.

\subsubsection{Technische Validierung der Android-Applikation}
Die technische Validierung der Android-Applikation erfolgte sowohl über
automatisierte Tests als auch durch kontinuierliches manuelles Testen
während der Entwicklung. Ziel war es, Funktionalitäten schrittweise
abzusichern und Fehler möglichst frühzeitig zu erkennen.

Unit Tests wurden im gesamten Projekt eingesetzt Dabei handelte es sich meist um kleine,
schnell ausführbare Tests, die während der Entwicklung laufend genutzt
wurden. Gerade nach Änderungen am Code waren diese Tests hilfreich, da sie rasch
aufzeigen konnten, ob bestehende Funktionalitäten weiterhin korrekt
arbeiteten. So konnten Fehler früh erkannt werden, ohne die App jedes Mal
manuell testen zu müssen.  

Eine vollständige Testabdeckung aller Module und Features konnte im
gegebenen Projektrahmen nicht umgesetzt werden, da die verfügbare
Projektzeit begrenzt war und ein grosser Teil der Aufwände in die
Entwicklung der Applikation investiert wurde (siehe Anhang~\ref{anhang:zeit}). 
Die vorhandenen Tests wurden jeweils parallel zur Feature-Entwicklung
erstellt. Das Ergänzen weiterer Tests ist problemlos möglich, da die
bestehende Teststruktur modular aufgebaut ist und einfache Erweiterungen
zulässt.

Nach der Abgabe der Dokumentation wird bis zur Abschlusspräsentation
weiter am Projekt gearbeitet, mit einem stärkeren Fokus auf das Testing.
Mit dieser zusätzlichen Zeit wird angestrebt, die Testabdeckung auf
rund 80\,\% zu erhöhen.

\vspace{0.5em}

\begin{figure}[H]
    \centering
    \includegraphics[width=0.6\textwidth]{Fotos/testfolders.png}
    \caption{Teststruktur im Android Studio}
    \label{fig:teststruktur}
\end{figure}

\newpage

\subsubsection{Manuelles Testen}

Neben den automatisierten Tests wurde die Applikation während der Entwicklung
regelmässig manuell getestet. Der Grossteil dieser Tests erfolgte mit dem
Android-Emulator, da sich damit unterschiedliche Android-Versionen,
Displaygrössen und Konfigurationen effizient abdecken lassen.

Zusätzlich wurde die App auf ausgewählten physischen Geräten getestet,
konkret auf einem \textit{Google Pixel} sowie einem \textit{Samsung Galaxy S22}.
Da Android jedoch auf einer sehr grossen Vielfalt an Geräten und
Herstellerkonfigurationen läuft und nur eine begrenzte Anzahl an
Testgeräten zur Verfügung stand, konnte keine breite Abdeckung aller
Geräteklassen erreicht werden.

Dieses Vorgehen und die daraus resultierende Einschränkung bei den Tests auf
realen Geräten wurden im Projektverlauf mit dem Auftraggeber besprochen und
als ausreichend für den gegebenen Projektrahmen beurteilt.
(siehe Anhang~\ref{anhang:report30102025}).

\subsubsection{Eingesetzte Testframeworks}

Im Projekt kamen folgende Testframeworks zum Einsatz:

\begin{itemize}
    \item \textbf{JUnit 5}:  
    Basis-Framework für Unit- und einfache Integrationstests. Es wird im gesamten
    Projekt eingesetzt, um einzelne Funktionen, Klassen und ViewModels 
    zu validieren und deren Logik zu überprüfen.

    \item \textbf{MockK}:  
    Mocking-Framework für Kotlin. Es wird verwendet, um Abhängigkeiten wie
    Services oder Repositories zu mocken, beispielsweise beim Testen von
    Netzwerkzugriffen oder Datenlade-Logik ohne echte API-Aufrufe.

    \item \textbf{Espresso}:  
    Framework für UI- und Integrationstests im Android-Kontext. Es kommt bei
    einfachen Benutzerinteraktionen, Navigationsflüssen und grundlegenden
    UI-Validierungen zum Einsatz.
\end{itemize}


\subsection{Validierung der Anforderungen und Projektziele}

Die Validierung der fachlichen Anforderungen und Projektziele erfolgte iterativ
über den gesamten Projektverlauf hinweg. Die Entwicklung wurde sprint-basiert
organisiert, wobei alle Aufgaben und Features als Issues in GitLab erfasst und
verwaltet wurden.

Alle zwei Wochen fanden Meetings mit dem Auftraggeber statt, in denen der aktuelle
Projektstand vorgestellt, schriftlich dokumentiert und gemeinsam besprochen wurde.
(siehe Anhang~\ref{sec:protokolle-statusberichte}).
Dabei wurden umgesetzte Features überprüft, offene Punkte geklärt und die
Prioritäten für den nächsten Sprint festgelegt. Auf diese Weise konnte laufend
validiert werden, ob das Projekt in die gewünschte Richtung entwickelt wird.

Durch das Issue-Tracking in GitLab war jederzeit ersichtlich, welche Aufgaben
bereits abgeschlossen waren und welche noch offen standen. Am Ende des Projekts
sind alle geplanten Issues geschlossen, was bedeutet, dass die vereinbarten
Features umgesetzt wurden.

\subsection{Validierung des Nutzens}

Im Rahmen dieses Projekts wurde keine Validierung mit Studierenden durchgeführt, 
um zu überprüfen, ob die App im Alltag tatsächlich als nützlich wahrgenommen 
wird oder einen konkreten Mehrwert bietet. Eine solche Nutzerstudie hätte den 
inhaltlichen Rahmen des Projekts deutlich überschritten und war daher
nicht Bestandteil des Auftrags.