\subsection{Mensa}

Studierenden soll der aktuelle Menüplan des
Mensa-Betreibers ZFV möglichst schnell und unkompliziert zur Verfügung stehen.
Da zum Zeitpunkt der Entwicklung unklar war, ob ZFV eine stabile oder offiziell
unterstützte API anbietet, wurde hier bewusst auf eine JSON-basierte Darstellung
verzichtet und stattdessen eine WebView-Lösung umgesetzt. Die technische Umsetzung
erfolgt über den \texttt{MensaModuleLoader} sowie eine einfache Compose-View, die
wie zuvor schon \texttt{News} und \texttt{Blog} den 
\texttt{WebViewScreen} einbettet, um die Webseite anzuzeigen.

Der ZFV stellt den Menüplan neben der regulären Webseite auch als \textit{iframe}
bereit. Ein \textit{iframe} erlaubt das Einbetten eines externen Webseitenabschnitts
innerhalb einer anderen Website, wodurch nur ein bestimmter Teil des Inhalts
geladen wird, statt der gesamten Seite. Diese Flexibilität wird im Backend genutzt:
Dort kann konfiguriert werden, ob die gesamte Mensa-Webseite oder lediglich der
eingebettete iframe-Inhalt angezeigt werden soll. Beide Varianten lassen sich
einfach über die Hinterlegung der jeweiligen URL steuern und erfordern keine
Anpassungen in der App selbst.

Die beiden Darstellungsoptionen sind hier zu sehen: \newline

\vspace{0.5cm}

\begin{figure}[H]
    \centering
    \begin{subfigure}[b]{0.23\textwidth}
        \centering
        \includegraphics[width=\textwidth]{Fotos/feature-screenshots/mensa_site.png}
        \caption{Ansicht gesamte Webseite}
        \label{fig:mensa_site}
    \end{subfigure}
    \hspace{1.2cm}
    \begin{subfigure}[b]{0.23\textwidth}
        \centering
        \includegraphics[width=\textwidth]{Fotos/feature-screenshots/mensa_iframe.png}
        \caption{Iframe-Ansicht}
        \label{fig:mensa_iframe}
    \end{subfigure}
    \caption{Screenshots des Mensa-Features}
    \label{fig:mensa_feature}
\end{figure}
