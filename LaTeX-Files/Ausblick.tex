\section{Reflexion} \label{sec:reflexion}

\subsection{Fazit AI-First Programmierung} \label{sec:ai-reflexion}

Der \textit{AI-First}-Ansatz hat uns während der gesamten Projektdauer begleitet.
Wir hatten bereits vor Projektbeginn Erfahrung im Umgang mit
KI-Tools, jedoch wurde künstliche Intelligenz in diesem Projekt so intensiv
eingesetzt wie zuvor noch nicht. Über die gesamte Laufzeit konnten dabei sowohl
positive als auch kritische Erfahrungen gesammelt werden.

Insgesamt überwiegen für uns klar die positiven Aspekte. Die Zeitersparnis durch
den Einsatz von AI war sehr gross und hat sich im Projektalltag deutlich
bemerkbar gemacht. Da wir beide bereits mehrere Jahre Erfahrung in der
Informatik haben und auch noch ohne AI-gestützte Entwicklung gearbeitet haben,
war der Vergleich für uns gut möglich. Das Projekt hat bestätigt, dass AI den
Entwicklungsprozess beschleunigt und viele alltägliche Aufgaben 
vereinfacht. Unsere Erwartungen an den Nutzen von AI wurden dabei grösstenteils
erfüllt.

Wichtig ist aus unserer Sicht jedoch, dass der Output einer AI stark vom Input
abhängt. Gute und präzise Prompts führen in der Regel auch zu brauchbaren Resultaten,
während vage oder ungenaue Eingaben oft zu weniger hilfreichen Antworten führen.
Aus diesem Grund möchten wir unsere Erfahrungen festhalten und weitergeben, da
wir den bewussten Einsatz von AI als zentral erachten und diese Erfahrungen weitergeben möchten. Auch
wenn sich die Möglichkeiten von AI sehr schnell weiterentwickeln, lassen sich
heute noch klare Stärken und Grenzen erkennen.

\subsubsection*{Stärken}
Aus unserer Sicht eignet sich AI besonders gut für folgende Aufgaben:
\begin{itemize}
    \item Analyse von bestehendem Code: AI kann sehr gut erklären, wie Code
    aufgebaut ist und wo sich welche Logik befindet, wodurch sich schnell ein
    Überblick gewinnen lässt.
    \item Übersetzung von bestehendem Code in andere Sprachen (z.\,B. Swift nach
    Kotlin), da der vorhandene Code als sehr präziser Prompt dient.
    \item Schreiben einzelner Funktionen oder Klassen, sofern diese nicht zu
    umfangreich sind.
    \item Umsetzung ähnlicher Features auf Basis bestehender Funktionalität
    (z.\,B. News $\rightarrow$ Blog), da vorhandener Code als gute Vorlage dient.
    \item Allgemeine Arbeitserleichterung im Vergleich zum klassischen Vorgehen
    (Problemsuche, Recherche im Internet, StackOverflow, Dokumentation).
\end{itemize}

\subsubsection*{Grenzen und Herausforderungen}
Trotz der vielen Vorteile gibt es aus unserer Sicht auch klare Einschränkungen:
\begin{itemize}
    \item Die Generierung von gut lesbarer und inhaltlich ausgewogener
    Dokumentation ist noch schwierig. Texte sind zwar nicht grundsätzlich
    schlecht, enthalten aber oft auffällige Füllwörter (beispielsweise
    „insbesondere, „vollständig“, „speziell“),
    die dem Leser auffallen. Zudem werden teilweise wichtige Details
    ausgelassen, die eigentlich relevant für das Verständnis wären.
    \item Das Generieren von Diagrammen funktioniert aus unserer Sicht noch nicht
    besonders gut. Zwar konnte Cursor vor allem Ablauf- und Sequenzdiagramme relativ
    häufig erzeugen, und diese ergaben teilweise sogar noch Sinn, da der reine
    Programmablauf abgebildet wird.
    \item Bei eher abstrakten oder sogenannten \glqq top-down\grqq{} Diagrammen,
    wie zum Beispiel Anwendungsfalldiagrammen, stösst die AI klar an ihre Grenzen.
    Hier müsste das Gesamtsystem wirklich verstanden werden, anstatt nur den
    Codefluss wiederzugeben. Unser Eindruck ist, dass die AI dafür aktuell noch
    nicht ausreichend weit ist.
    \item Beim Erstellen von UI-Layouts liefert AI zwar grobe Strukturen, das
    Ergebnis entspricht jedoch selten den Erwartungen. Unserer Erfahrung nach kann
     die AI nur schwer einschätzen, wie ein Layout am Ende tatsächlich auf dem Bildschirm aussieht (siehe ~\ref{anhang:ai-first-elements}).
    \item Clean-Code-Prinzipien werden nicht immer konsequent eingehalten.
    Teilweise entstehen doppelte Implementierungen oder ungünstige Strukturen,
    da der Gesamtzusammenhang nicht vollständig erfasst wird.
    \item Syntaxfehler werden nicht immer erkannt und müssen explizit korrigiert
    werden (siehe ~\ref{anhang:ai-first-elements}).
    \item Aufgrund veralteter Trainingsdaten werden teilweise ältere oder nicht
    mehr empfohlene Bibliotheken vorgeschlagen, die manuell aktualisiert werden
    müssen.
\end{itemize}

\subsubsection*{Gesamtfazit}
Aus unserer Sicht kann AI derzeit keine vollständige Applikation ohne fundierte
Programmierkenntnisse ersetzen. Diese sind weiterhin notwendig, um Fehler zu
beheben, Architekturentscheidungen zu treffen und eine saubere Struktur
zu gewährleisten. Dennoch ist AI ein Werkzeug, auf das moderne
Softwareentwicklung kaum mehr verzichten kann. Die Arbeitserleichterung ist
gross, birgt aber auch die Gefahr, sich zu stark darauf zu verlassen. Ein
bewusster Einsatz ist unserer Meinung nach daher entscheidend.

\subsection{Team-Fazit} \label{sec:team-fazit}
\subsection{Persönliche Reflexion Raphael}
\subsection{Persönliche Reflexion Samuel}

\section{Ausblick}
\subsection{Persönlicher Ausblick} \label{sec:ausblick}
\subsection{Mögliche weiterführende Arbeiten} \label{sec:weiterfuehrende-arbeiten}