\section{Reflexion} \label{sec:reflexion}

\subsection{Fazit AI-First Programmierung} \label{sec:ai-reflexion}

Der \textit{AI-First}-Ansatz hat uns während der gesamten Projektdauer begleitet.
Wir hatten bereits vor Projektbeginn Erfahrung im Umgang mit
KI-Tools, jedoch wurde künstliche Intelligenz in diesem Projekt so intensiv
eingesetzt wie zuvor noch nicht. Über die gesamte Laufzeit konnten dabei sowohl
positive als auch kritische Erfahrungen gesammelt werden.

Insgesamt überwiegen für uns klar die positiven Aspekte. Die Zeitersparnis durch
den Einsatz von AI war sehr gross und hat sich im Projektalltag deutlich
bemerkbar gemacht. Da wir beide bereits mehrere Jahre Erfahrung in der
Informatik haben und auch noch ohne AI-gestützte Entwicklung gearbeitet haben,
war der Vergleich für uns gut möglich. Das Projekt hat bestätigt, dass AI den
Entwicklungsprozess beschleunigt und viele alltägliche Aufgaben 
vereinfacht. Unsere Erwartungen an den Nutzen von AI wurden dabei grösstenteils
erfüllt.

Wichtig ist aus unserer Sicht jedoch, dass der Output einer AI stark vom Input
abhängt. Gute und präzise Prompts führen in der Regel auch zu brauchbaren Resultaten,
während vage oder ungenaue Eingaben oft zu weniger hilfreichen Antworten führen.
Aus diesem Grund möchten wir unsere Erfahrungen festhalten und weitergeben, da
wir den bewussten Einsatz von AI als zentral erachten und diese Erfahrungen weitergeben möchten. Auch
wenn sich die Möglichkeiten von AI sehr schnell weiterentwickeln, lassen sich
heute noch klare Stärken und Grenzen erkennen.

\subsubsection*{Stärken}
Aus unserer Sicht eignet sich AI besonders gut für folgende Aufgaben:
\begin{itemize}
    \item Analyse von bestehendem Code: AI kann sehr gut erklären, wie Code
    aufgebaut ist und wo sich welche Logik befindet, wodurch sich schnell ein
    Überblick gewinnen lässt.
    \item Übersetzung von bestehendem Code in andere Sprachen (z.\,B. Swift nach
    Kotlin), da der vorhandene Code als sehr präziser Prompt dient.
    \item Schreiben einzelner Funktionen oder Klassen, sofern diese nicht zu
    umfangreich sind.
    \item Umsetzung ähnlicher Features auf Basis bestehender Funktionalität
    (z.\,B. News $\rightarrow$ Blog), da vorhandener Code als gute Vorlage dient.
    \item Allgemeine Arbeitserleichterung im Vergleich zum klassischen Vorgehen
    (Problemsuche, Recherche im Internet, StackOverflow, Dokumentation).
\end{itemize}

\subsubsection*{Grenzen und Herausforderungen}
Trotz der vielen Vorteile gibt es aus unserer Sicht auch klare Einschränkungen:
\begin{itemize}
    \item Die Generierung von gut lesbarer und inhaltlich ausgewogener
    Dokumentation ist noch schwierig. Texte sind zwar nicht grundsätzlich
    schlecht, enthalten aber oft auffällige Füllwörter (beispielsweise
    „insbesondere, „vollständig“, „speziell“),
    die dem Leser auffallen. Zudem werden teilweise wichtige Details
    ausgelassen, die eigentlich relevant für das Verständnis wären.
    \item Das Generieren von Diagrammen funktioniert aus unserer Sicht noch nicht
    besonders gut. Zwar konnte Cursor vor allem Ablauf- und Sequenzdiagramme relativ
    häufig erzeugen, und diese ergaben teilweise sogar noch Sinn, da der reine
    Programmablauf abgebildet wird.
    \item Bei eher abstrakten oder sogenannten \glqq top-down\grqq{} Diagrammen,
    wie zum Beispiel Anwendungsfalldiagrammen, stösst die AI klar an ihre Grenzen.
    Hier müsste das Gesamtsystem wirklich verstanden werden, anstatt nur den
    Codefluss wiederzugeben. Unser Eindruck ist, dass die AI dafür aktuell noch
    nicht ausreichend weit ist.
    \item Beim Erstellen von UI-Layouts liefert AI zwar grobe Strukturen, das
    Ergebnis entspricht jedoch selten den Erwartungen. Unserer Erfahrung nach kann
     die AI nur schwer einschätzen, wie ein Layout am Ende tatsächlich auf dem Bildschirm aussieht (siehe ~\ref{anhang:ai-first-elements}).
    \item Clean-Code-Prinzipien werden nicht immer konsequent eingehalten.
    Teilweise entstehen doppelte Implementierungen oder ungünstige Strukturen,
    da der Gesamtzusammenhang nicht vollständig erfasst wird.
    \item Syntaxfehler werden nicht immer erkannt und müssen explizit korrigiert
    werden (siehe ~\ref{anhang:ai-first-elements}).
    \item Aufgrund veralteter Trainingsdaten werden teilweise ältere oder nicht
    mehr empfohlene Bibliotheken vorgeschlagen, die manuell aktualisiert werden
    müssen.
\end{itemize}

\subsubsection*{Gesamtfazit}
Aus unserer Sicht kann AI derzeit keine vollständige Applikation ohne fundierte
Programmierkenntnisse ersetzen. Diese sind weiterhin notwendig, um Fehler zu
beheben, Architekturentscheidungen zu treffen und eine saubere Struktur
zu gewährleisten. Dennoch ist AI ein Werkzeug, auf das moderne
Softwareentwicklung kaum mehr verzichten kann. Die Arbeitserleichterung ist
gross, birgt aber auch die Gefahr, sich zu stark darauf zu verlassen. Ein
bewusster Einsatz ist unserer Meinung nach daher entscheidend.

\subsection{Team-Fazit} \label{sec:team-fazit}
\subsubsection*{Zusammenarbeit im Team}
Die Zusammenarbeit im Team war sehr gut. Beide haben durchgehend fleissig am
Projekt gearbeitet. Die Kommunikation funktionierte ebenfalls sehr gut, da wir
bereits mehrere Hochschulprojekte gemeinsam umgesetzt haben und die Arbeitsweise
und Stärken des jeweils anderen gut kennen. Dadurch konnten Aufgaben effizient
aufgeteilt werden und es war jederzeit klar, wer sich um welchen Bereich kümmert.

\subsubsection*{Projekt und Umsetzung}
Im Projekt wurde insgesamt sehr viel umgesetzt und entsprechend auch viel Code
geschrieben. Zu Beginn war das Projekt technisch relativ komplex, insbesondere
durch Themen wie Multi-Tenant, Feature-Struktur und Backend-Anbindung. Nach einer
kurzen Einarbeitungsphase konnte jedoch gut damit gearbeitet werden. Rückblickend
hätten wir zu Projektbeginn eventuell noch ein paar zusätzliche Fragen an den
Auftraggeber stellen können, um gewisse Punkte früher zu klären.

Der Projektstand war über die gesamte Laufzeit gut, es wurde kontinuierlich am
Projekt gearbeitet und (fast) alle benötigten Features konnten umgesetzt werden.
Die Dokumentation mit LaTeX empfanden wir als sehr praktisch, da sie (ähnlich
wie Programmcode) sauber strukturiert, versioniert und direkt im Repository
abgelegt werden konnte.

Positiv hervorzuheben ist auch der Umgang mit Änderungen im Projektumfang.
Anpassungen wie JSON-basierte Views, API-Versionierung oder die Umsetzung der
Raumsuche als PDF konnten während des Projektes aufgenommen und integriert werden. Auch wenn
solche Änderungen im agilen Umfeld normal sind, sind wir der Meinung, dass dies im
Rahmen der WIPRO gut umgesetzt wurde.

Nicht ganz zufrieden sind wir mit dem aktuellen Stand der Testabdeckung. Zwar
wollten wir diese im Vergleich zu früheren Projekten bewusst verbessern, jedoch
war gegen Projektende kaum noch Zeit dafür vorhanden. An diesem Punkt wird bis zur
Abschlusspräsentation jedoch noch weiter gearbeitet.\newpage

\subsubsection*{Projektplanung}
Die Projektplanung war zu Beginn nicht ganz einfach, da noch unklar war, wo genau
mehr Zeit investiert werden muss. Der agile Ansatz hat sich hier klar bewährt.
Die Projektmeetings mit dem Auftraggeber waren stets sehr hilfreich und konnten bei
offenen Fragen oder Problemen weiterhelfen. Rückblickend wären wöchentliche Meetings
teilweise sogar sinnvoll gewesen, da oft viele Punkte zu diskutieren waren und
frühere Klärungen hilfreich gewesen wären.

Die Arbeit mit GitLab-Issues empfanden wir als sehr praktisch, da der Projektstand
jederzeit sichtbar war. Dieses Vorgehen würden wir in zukünftigen Projekten
wieder so wählen. Das Risikomanagement war aus unserer Sicht in Ordnung, da wir uns
den Risiken während des Projekts stets bewusst waren. Teilweise war es jedoch etwas
mühsam, da über längere Zeit unklar war, ob gewisse Features (z.\,B. der
Parkplatzzähler) überhaupt noch benötigt werden.

Das agile Arbeiten insgesamt empfanden wir als sehr passend, insbesondere da wir
beide berufsbegleitend studieren und nebenbei arbeiten. Es war nicht immer möglich,
täglich Meetings abzuhalten, weshalb es sehr praktisch war, Aufgaben über Issues
zu übernehmen und selbstständig abzuarbeiten.

\subsubsection*{Lerneffekt}
Der Lerneffekt aus dem Projekt war für uns beide sehr gross. Zwar haben wir bereits
die Module \textit{MobPro} und \textit{MobLab} besucht, konnten hier aber besonders
von der Grösse und Komplexität der Applikation profitieren. Die modulare Architektur
eignete sich sehr gut, um ein sauberes App-Design zu lernen und zu vertiefen.

Sehr spannend war zudem die intensive Arbeit mit AI-Tools. Dabei konnten wir viel
darüber lernen, wie diese effizient eingesetzt werden können, um im Gesamtkontext
Zeit zu sparen. Auch das Multi-Tenant-Konzept in Kombination mit der
Backend-Anbindung war sehr interessant, da hier bereits eine saubere und gut
durchdachte Lösung bestand, an der weitergearbeitet werden konnte.

Zusätzlich konnten wir unsere Kenntnisse in Jetpack Compose, moderner
Android-Architektur sowie unseren realistischen Blick auf die Stärken und Grenzen
von AI weiter vertiefen.


\subsection{Persönliche Reflexion Raphael}
Rückblickend bin ich mit meinem Einsatz im Projekt zufrieden. Ich habe
durchgehend aktiv am Projekt mitgearbeitet, konnte viele Aufgaben selbstständig
umsetzen und dabei sehr viel lernen. Besonders die Arbeit an einer
umfangreichen und technisch anspruchsvollen Applikation war für mich sehr
lehrreich.

Der \textit{AI-First}-Ansatz war für mich persönlich eines der spannendsten
Elemente des Projekts. Der Umgang mit verschiedenen AI-Tools und deren gezielter
Einsatz im Entwicklungsprozess war sehr interessant. Mit der Zeit wurde ich
deutlich besser darin, sinnvolle und präzise Prompts zu formulieren, was sich
direkt auf die Qualität der Resultate ausgewirkt hat. Dieser Lernprozess war
für mich klar spürbar und wird mir auch in zukünftigen Projekten weiterhelfen.

Was ich rückblickend anders machen würde, ist zu Projektbeginn mehr Zeit in das
vollständige Verständnis des Projekts zu investieren, bevor mit der eigentlichen
Umsetzung gestartet wird. Zu Beginn habe ich einige Dinge zu schnell umgesetzt,
die später nochmals überarbeitet werden mussten. Eine gründlichere
Einarbeitungsphase hätte hier vermutlich Zeit gespart.

Elemente, die ich in zukünftigen Projekten wieder genauso umsetzen würde, sind
der AI-First-Ansatz, die regelmässigen Statusberichte und Meetings sowie das
strukturierte Vorgehen bei der Abarbeitung der Issues. Diese Arbeitsweise hat
sich für mich bewährt und wesentlich zum guten Projektfortschritt beigetragen.

\subsection{Persönliche Reflexion Samuel}

\newpage
\section{Ausblick}
\subsection{Persönlicher Ausblick} \label{sec:ausblick}

Das Projekt ist mit der Abgabe der Dokumentation noch nicht vollständig
abgeschlossen. Aus persönlichem Interesse wird der Programmcode in der Zeit bis
zur Abschlusspräsentation noch leicht überarbeitet und refactored, um kleinere
Unschönheiten (auch im Zusammenhang mit dem AI-First-Ansatz) auszubessern.
Zudem wird weiter an der Testabdeckung gearbeitet, da zwischen Abgabe der
Dokumentation und der Abschlusspräsentation noch knapp einen Monat Zeit bleibt.

Am 15.~Januar 2026 findet die Abschlusspräsentation statt. Anschliessend soll die App
an den Auftraggeber Jürg Nietlispach übergeben werden, der danach die
Verantwortung für die Weiterentwicklung und den Betrieb der Applikation
übernimmt.

\subsection{Mögliche weiterführende Arbeiten} \label{sec:weiterfuehrende-arbeiten}
Als weiterführende Arbeiten ist der Weg offen, die Applikation schrittweise mit
neuen Features zu ergänzen. Durch den modularen Aufbau lässt sich die App gut
erweitern, ohne bestehende Funktionalitäten stark anzupassen.

Eine mögliche Weiterentwicklung wäre, Studierende gezielt zu befragen, welche
weiteren Funktionen sie als sinnvoll erachten. Denkbar wären beispielsweise
Erweiterungen wie eine Anzeige von Noten aus MyCampus, ein Credits-Rechner oder
zusätzliche Modulbeschreibungen.

Solche Funktionen könnten mit dem aktuellen technischen Aufbau sowohl auf iOS
als auch auf Android gut ergänzt werden und den Nutzen der App für Studierende
weiter erhöhen.