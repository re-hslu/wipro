\section*{Einleitung}

Das Ziel der Entwicklung war es, eine modulare und erweiterbare Applikation zu schaffen. 
Dazu wurde auf die bestehende iOS-Projektstruktur aufgebaut und das Android-Projekt analog dazu umgesetzt. 

Im Wesentlichen wurde die Anwendung in die beiden Hauptmodule \texttt{common} und \texttt{features} gegliedert:
\begin{itemize}
    \item \textbf{common}: Enthält alle allgemeinen und wiederverwendbaren Komponenten, wie z.\,B. UI-Elemente, Utility-Klassen und grundlegende Architekturbausteine.
    \item \textbf{features}: Beinhaltet die einzelnen Funktionsmodule der App, die jeweils auf spezifische Anwendungsbereiche oder Features ausgerichtet sind.
\end{itemize}

Dabei gilt die Abhängigkeitsregel, dass \texttt{features}-Module auf Inhalte aus \texttt{common} zugreifen dürfen, 
nicht jedoch umgekehrt – um zirkuläre Abhängigkeiten und eine enge Kopplung zu vermeiden.

\vspace{1em}

\begin{figure}[H]
    \centering
    \includegraphics[width=0.375\textwidth]{Fotos/projektexplorer_androidstudio.png}
    \caption{Projektstruktur im Android Studio}
    \label{fig:projektstruktur_androidstudio}
\end{figure}