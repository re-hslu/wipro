\subsection{Timetable}

Das Feature \textit{Timetable} dient dazu, den persönlichen Stundenplan der
Studierenden direkt in der HSLU-App anzuzeigen. Die Grundidee ist gleich wie bei der
iOS-App: Die Applikation greift nicht direkt auf \textit{MyCampus} zu, sondern
setzt voraus, dass die Nutzer:innen ihren Studienkalender bereits über
MyCampus per Link in ihre lokale Kalender-App eingebunden haben. Eine direkte
MyCampus-Schnittstelle ist aktuell nicht vorgesehen, daher bleibt der lokal
installierte Android-Kalender die einzige Datenquelle.

\subsubsection*{Zugriff auf den Android-Kalender}

Android stellt über das \texttt{CalendarContract}-API einen strukturierten
Zugriff auf Kalender und Termine bereit \parencite{google_android_nodate}.  
Die App kann damit:
\begin{itemize}
    \item verfügbare Kalender des Geräts auslesen,
    \item die Ereignisse eines gewählten Kalenders für einen bestimmten Zeitraum
          laden,
    \item sowie alle Einträge nach Start- und Endzeit filtern.
\end{itemize}

Damit dieses Feature funktioniert, muss der Nutzer der App die
\texttt{READ\_CALENDAR}-Berechtigung erteilen. Beim Öffnen des Features erscheint
die Abfrage für die Berechtigung automatisch. Anschliessend kann in der \newline
\texttt{TimetableCalendarSwitcher}-View ein Kalender ausgewählt werden. Die Wahl
wird in den \textit{SharedPreferences} gespeichert, sodass der Nutzer sie nicht
bei jedem App-Start erneut treffen muss. Änderungen können jederzeit über die
Einstellungen der App vorgenommen werden.

Die eigentliche Verarbeitung der Kalendereinträge erfolgt über den
\texttt{CommonCalendarDataLoader}, der mithilfe von
\texttt{ContentResolver.query()} die Einträge über
\texttt{CalendarContract.Instances} abruft. Aus den Resultaten werden Objekte
vom Typ \texttt{LectureDTO} instanziert, welche unter anderem Name, Notizen,
Raum, Start-, Endzeit sowie diverse abgeleitete Informationen enthalten
(z.\,B. Kurzname, formattiertes Datum, ILIAS-Link). Damit das Erstellen der 
\texttt{LectureDTO}-Objekte korrekt abläuft, müssen die Kalendereinträge vom
Format so im Kalender vorhanden sein, wie sie auch von MyCampus bereitgestellt
werden.

\subsubsection*{Darstellung in der App}

Nach erfolgreichem Laden wird der Stundenplan für die nächsten sieben Tage
aufbereitet und in einer strukturierten Übersicht dargestellt. Die
Kalendereinträge werden gruppiert nach:
\begin{itemize}
    \item \textbf{Heute},
    \item \textbf{Morgen},
    \item \textbf{Datum} (für spätere Tage).
\end{itemize}

Jede Tagesgruppe wird in einer eigenen
Sektion dargestellt, wie in der \texttt{TimetableListView} ersichtlich.
Innerhalb der Sektionen werden die Details pro Vorlesung angezeigt, darunter
Kurzname, vollständiger Titel, Zeitfenster, Raum sowie weiterführende
Informationen. Bei nicht vorhandenen zukünftigen Terminen wird automatisch die
\texttt{TimetableEmptyView} angezeigt.

\subsubsection*{Erkennung von Raumänderungen}

Ein praktischer Zusatznutzen dieses Features ist die Erkennung von
Raumänderungen gegenüber der Vorwoche. Die App vergleicht frühere und aktuelle
Einträge und markiert Vorlesungen visuell, wenn sich der Raum geändert hat.
Dies verhindert, dass Studierende versehentlich in den falschen Raum gehen und
erhöht die Alltagstauglichkeit des Features.

\newpage

\subsubsection*{Ergebnis}

Das Timetable-Feature bietet eine kompakte und alltagstaugliche Übersicht der
kommenden Vorlesungen, gefiltert auf die relevanten Tage und ohne Ablenkung
durch andere private Termine, wie sie in der Standard-Kalender-App erscheinen.
Studierende erhalten damit einen schnellen, Zugang zu den
Hochschulterminen direkt innerhalb der mobilen Applikation, ohne App- oder
Kontextwechsel.

\vspace{0.5cm}

\begin{figure}[H]
    \centering
    \begin{subfigure}[b]{0.23\textwidth}
        \centering
        \includegraphics[width=\textwidth]{Fotos/feature-screenshots/timetable_calendar.png}
        \caption{Eintrag in Kalender}
        \label{fig:timetable_calendar}
    \end{subfigure}
    \hspace{0.3cm}
    \begin{subfigure}[b]{0.23\textwidth}
        \centering
        \includegraphics[width=\textwidth]{Fotos/feature-screenshots/timetable_overview.png}
        \caption{Nächste Module}
        \label{fig:timetable_overview}
    \end{subfigure}
    \hspace{0.3cm}
    \begin{subfigure}[b]{0.23\textwidth}
        \centering
        \includegraphics[width=\textwidth]{Fotos/feature-screenshots/timetable_changes.png}
        \caption{Raumänderungen}
        \label{fig:timetable_changes}
    \end{subfigure}
    \caption{Screenshots des Kalender-Features}
    \label{fig:timetable_feature}
\end{figure}

\vspace{0.5cm}

\subsubsection{Timetable-Widget}

Zusätzlich zur Stundenplan-Ansicht innerhalb der App bietet das \textit{Timetable}-Feature
ein Home-Screen-Widget an, welches die nächsten Vorlesungen direkt auf dem
Startbildschirm des Geräts anzeigt. Das Widget ist mit Jetpack
\textit{Glance} umgesetzt und nutzt damit die Compose-basierte
Widget-Architektur von Android \parencite{google_android_nodate-1}.

\subsubsection*{Registrierung und Integration}
Das Widget wird im \texttt{AndroidManifest.xml} registriert. 
Der \texttt{TimetableWidgetReceiver} reagiert auf das System-Event 
\texttt{APPWIDGET\_UPDATE} und verweist über Meta-Daten auf die 
Widget-Konfiguration \newline (\texttt{timetable\_widget\_info.xml}). 
Diese definiert unter anderem die Grösse und das automatische 
Update-Intervall (30 Minuten) des Widgets. 
Nach der Installation erscheint das Widget automatisch 
in der Widget-Auswahl des Systems.

\newpage

Ein Beispiel der Registrierung im \texttt{AndroidManifest.xml}:
\begin{lstlisting}[language=XML, caption={Auszug aus AndroidManifest.xml}]
<application>
    <receiver
        android:name=".widget.TimetableWidgetReceiver"
        android:exported="true">
        <intent-filter>
            <action android:name="android.appwidget.action.APPWIDGET_UPDATE" />
        </intent-filter>
        <meta-data
            android:name="android.appwidget.provider"
            android:resource="@xml/timetable_widget_info" />
    </receiver>
</application>
\end{lstlisting}

\subsubsection*{Widget-Architektur}
Für die Umsetzung wurde bewusst Jetpack Glance verwendet, da es die moderne,
Compose-orientierte Alternative zum klassischen \texttt{AppWidgetProvider} ist.
Die Struktur im Code ist wie folgt aufgebaut:
\begin{itemize}
    \item \textbf{TimetableWidgetReceiver}: Einstiegspunkt für das System,
          leitet Events an das Widget weiter.
    \item \textbf{TimetableWidget}: Lädt die Daten und stellt über
          \texttt{provideContent\{\}} den UI-Baum bereit.
    \item \textbf{TimetableWidgetContent}: Compose-basierte Darstellung des
          Widgets, inklusive Fehler-, Leer- und Erfolgszuständen.
\end{itemize}

Die Daten werden über den \texttt{TimetableWidgetDataProvider} geladen. Da Widgets
in einem separaten Prozess laufen und kein Dependency-Injection-Framework wie
Hilt verfügbar ist, erfolgt der Kalenderzugriff direkt über \newline \texttt{CalendarContract}
sowie die gespeicherten Einstellungen aus den \textit{SharedPreferences}.
Das Widget zeigt bis zu vier kommende Vorlesungen des aktuellen Tages an und blendet
bereits abgelaufene Termine automatisch aus.

Da Widgets ebenfalls keinen Zugriff auf \texttt{MaterialTheme} haben, wird die
Farbe manuell aus den App-Ressourcen geladen. Über
\texttt{getIdentifier()} wird der Farb-Resource-Name ermittelt und mit
\texttt{ContextCompat.getColor()} in ein Compose-\texttt{Color}-Objekt umgewandelt,
sodass das Widget trotzdem die definierte HSLU-Farbe verwenden kann.

\subsubsection*{UI-Konzept}
Die grafische Darstellung ist kompakt gehalten, um auf der kleinen
Widget-Flächen die nötigsten Informationen darzustellen. Je nach Zustand zeigt
das Widget:
\begin{itemize}
    \item eine Fehlermeldung (z.\,B. fehlende Kalenderberechtigung),
    \item die nächsten Vorlesungen des Tages inklusive Zeit, Kurzname und Raum,
    \item oder einen leeren Zustand (\enquote{Keine Vorlesungen heute}).
\end{itemize}

\subsubsection*{Ergebnis}
Das Timetable-Widget erweitert das Feature um eine praktische Ansicht, damit 
die Nutzer:innen den Stundenplan jederzeit einsehen können, ohne die App öffnen zu müssen.

\newpage
\vspace{1.5cm}

\begin{figure}[H]
    \centering
    \begin{subfigure}[b]{0.23\textwidth}
        \centering
        \includegraphics[width=\textwidth]{Fotos/feature-screenshots/widget_2.png}
        \caption{Zwei Vorlesungen}
        \label{fig:widget_2}
    \end{subfigure}
    \hspace{0.3cm}
    \begin{subfigure}[b]{0.23\textwidth}
        \centering
        \includegraphics[width=\textwidth]{Fotos/feature-screenshots/widget_0.png}
        \caption{Keine Vorlesungen}
        \label{fig:widget_0}
    \end{subfigure}
    \caption{Screenshots des Kalender-Widgets}
    \label{fig:timetable_widget}
\end{figure}