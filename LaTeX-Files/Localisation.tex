\subsection{Lokalisierung}

Die App ist so konzipiert, dass sie in mehreren Sprachen verfügbar ist
(aktuell Deutsch und Englisch). Eine grundlegende Lokalisierungslogik war
bereits im ursprünglichen Android-Projekt vorhanden und wurde im Rahmen
dieses Projekts weiterentwickelt.

Die Lokalisierung basiert weiterhin auf dem standardmässigen
Android-Resource-System. (Android wählt automatisch die passende strings.xml 
basierend auf dem System-Locale) 
Dieses System wurde zusätzlich so ergänzt, dass auch
dynamisch geladene Inhalte (beispielsweise Menüpunkte aus dem
Bootstrapping-Prozess) zur Laufzeit korrekt lokalisiert dargestellt
werden können.

\subsubsection*{Konzept}
\begin{itemize}
    \item Die Sprachauswahl erfolgt automatisch über die Systemeinstellung
    des Geräts; eine separate Auswahl der App-Sprache ist nicht vorgesehen.
    \item Es wird ein ressourcenbasiertes Lokalisierungskonzept verwendet:
    Deutsch ist als Standardsprache im Ordner \texttt{values/} hinterlegt,
    Englisch im Ordner \texttt{values-en/}. Diese Ordner existieren jeweils
    pro Feature-Modul.
    \item Die Verwendung von UI-Strings erfolgt Android-typisch über
    \texttt{stringResource(R.string.\dots)} in Compose beziehungsweise über
    \texttt{context.getString(\dots)}.
\end{itemize}

\hspace{0.3cm}

\begin{lstlisting}[language=XML, caption={Default (Deutsch) und Englisch in \texttt{strings.xml}}]
<!-- values/strings.xml -->
<string name="Roomsearch_NavItem">Raumsuche</string>
<string name="Settings_NavItem">Einstellungen</string>

<!-- values-en/strings.xml -->
<string name="Roomsearch_NavItem">Roomsearch</string>
<string name="Settings_NavItem">Settings</string>
\end{lstlisting}

\subsubsection*{Übersetzung dynamischer Inhalte}
Für dynamisch geladene Inhalte, wie beispielsweise Menüpunkte aus
Remote-DTOs (siehe Abschnitt~\ref{sec:featureappinit}), wird die Sprache zur Laufzeit 
anhand der aktuellen Systemsprache bestimmt. Die Auswahl erfolgt über \newline
\texttt{Locale.getDefault()} und ordnet die passenden Sprachfelder zu:

\hspace{0.3cm}

\begin{lstlisting}[language=Java, caption={Auswahl der passenden Sprachvariante}]
val label: String
    get() = when (Locale.getDefault().language.lowercase(Locale.getDefault())) {
        "de" -> LabelDE ?: LabelEN ?: "Unbenannt"
        else -> LabelEN ?: LabelDE ?: "Unnamed"
    }
\end{lstlisting}

(Es wird zuerst versucht, die aktuell gesetzte Systemsprache zu
verwenden. Ist für diese keine Übersetzung verfügbar, wird automatisch
auf die jeweils andere unterstützte Sprache zurückgegriffen. Sollte
auch diese nicht vorhanden sein, wird ein Fallback-Wert
(\textit{z.\,B.} \texttt{'Unnamed'}) verwendet.)

Damit sind alle Inhalte der App
vollständig auf Deutsch und Englisch verfügbar. Weitere Sprachen könnten
bei Bedarf einfach ergänzt werden, indem zusätzliche
\texttt{strings.xml}-Dateien angelegt werden.