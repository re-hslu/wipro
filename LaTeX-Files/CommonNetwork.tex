\subsection{Netzwerk}
Die gesamte iOS Netzwerkarchitektur wurde auf Android migriert. Dabei wurde versucht eine identische Architektur beizubehalten und gleiche Methodennamen sowie API-Struktur zu verwenden.
Weiter wurden auch die Patterns konsistent über beide Plattformen gehalten.
Es wurden moderne Android best Practices verwendet wie Kotlin Coroutines, StateFlow für reaktives UI, Hilt Dependency Injection und eine Type-safe API mit Compile-Time-Checks.
Sowohl wurde auch hier der Multitenant Ansatz berücksichtigt und je nach build Variants werden die korrekten URLs und Tokens verwendet, somit konnten Redundanzen verhindert werden.

\subsubsection{Architektur Übersicht}

\subsubsection{Kern-Komponenten}
\begin {table}[ h ] 
    \centering 
    \begin {tabular}{| c | c | c |} 
        \hline % Horizontale Linie
        Komponente & iOS & Android \\ 
        \hline % Horizontale Linie
        NetzwerkMonitor & NWPathMonitor & ConnectivityManager \\
        \hline % Horizontale Linie
        CommonNetworkService & async/await & suspend fun \\
        \hline % Horizontale Linie
        NetworkConfig & struct & data class \\
        \hline % Horizontale Linie
        UrlConstants & static let & object \\
        \hline % Horizontale Linie
        Dependency Injection & EnvironmentObject & Hilt \\
        \hline % Horizontale Linie
        State Management & @Published & StateFlow \\
        \hline % Horizontale Linie
    \end {tabular}
    \label { tab : kernkomponenten } % Label fuer Referenzierung
\end {table}

\subsubsection{Tenant-Konfiguration}
\begin {table}[ h ] 
    \centering 
    \begin {tabular}{| c | c | c | c |} 
        \hline % Horizontale Linie
        Tenant & Base URL & Client Token & Build Variant \\ 
        \hline % Horizontale Linie
        HSLU I & hslui.mobile-hslu.ch & xxxxxxxx-xxxx-xxxx-xxxx-xxxxxxxxxxxx & hslui \\
        \hline % Horizontale Linie
        HSLU TA & hsluta.mobile-hslu.ch & xxxxxxxx-xxxx-xxxx-xxxx-xxxxxxxxxxxx & hsluta \\
        \hline % Horizontale Linie
    \end {tabular}
    \label { tab : tenantkonfiguration } % Label fuer Referenzierung
\end {table}

\subsubsection{Datei-Struktur}
\dirtree{%
.1 Common/Infrastructure/src/main/java/ch/hslu/mobileapps/infrastructure/.
.2 network/.
.3 CommonNetworkService.kt .
.3 NetworkMonitor.kt
.3 NetworkConfig.kt
.2 config/.
.3 TenantUrlProvider.kt.
.2 di/.
.3 NetworkModule.kt.
.1 app/src/hslui/java/ch/hslu/i/.
.2 config/.
.3 UrlConstants.kt.
.3 TenantUrlProviderImpl.kt.
.2 di/.
.3 HsluiNetworkModule.kt.
.1 app/src/hsluta/java/ch/hslu/ta/.
.2 config/.
.3 UrlConstants.kt.
.3 TenantUrlProviderImpl.kt.
.2 di/.
.3 HslutaNetworkModule.kt.
}

\subsubsection{Tests}
\textbf{Unittests}\\
\textbf{Integrationtests}\\