\subsubsection{Network}  \label{sec:commonnetwork}

Die Klasse CommonNetworkService stellt die zentrale Netzwerkschicht der Anwendung dar und bietet eine einheitliche Schnittstelle für alle Netzwerkoperationen. 
Es abstrahiert die Komplexität der Netzwerkkommunikation und bietet typsichere Methoden für HTTP-Requests, JSON-Parsing und Datei-Downloads. 
Die gesamte iOS Netzwerkarchitektur wurde auf Android migriert, wobei eine identische Architektur beibehalten wurde und gleiche Methodennamen sowie API-Struktur verwendet werden.
Es wurden moderne Android Best Practices verwendet wie Kotlin Coroutines, StateFlow für reaktives UI, Hilt Dependency Injection und eine Type-safe API mit Compile-Time-Checks.
Der Multitenant-Ansatz wurde berücksichtigt, sodass je nach Build-Varianten die korrekten URLs und Tokens verwendet werden, wodurch Redundanzen verhindert werden konnten.

\subsubsection*{Ziel und Motivation}

Das Hauptziel besteht darin, eine zentrale, wiederverwendbare Netzwerkschicht zu schaffen, die von allen Feature-Modulen verwendet werden kann, 
ohne dass jedes Modul seine eigene Netzwerk-Implementierung benötigt. Durch die Verwendung von Kotlin Coroutines und suspend-Funktionen wird eine moderne, 
asynchrone API bereitgestellt, die den Main-Thread nicht blockiert und eine reaktive Programmierung ermöglicht.

Die Verwendung von Hilt für Dependency Injection ermöglicht eine saubere Trennung von Konfiguration und Implementierung, 
während der Multitenant-Ansatz sicherstellt, dass verschiedene Build-Varianten (hslui, hsluta) automatisch die korrekten URLs und Tokens verwenden.

\subsubsection*{Umsetzung / Funktionsweise}

Die Implementierung basiert auf Android's HTTP-Client-Bibliotheken und Kotlin Coroutines. 
Der \newline \texttt{CommonNetworkService} ist die zentrale Service-Klasse, die alle Netzwerkoperationen kapselt:

\begin{lstlisting}[language=Kotlin, caption={Klasse CommonNetworkService}]
class CommonNetworkService(
    private val httpClient: OkHttpClient,
    private val tenantUrlProvider: TenantUrlProvider
) {
    suspend fun getAsJsonObject(url: String): JsonObject?
    // ...
}
\end{lstlisting}

Der \texttt{NetworkMonitor} überwacht die Netzwerkverbindung und stellt Informationen über die aktuelle Verbindungsqualität bereit. 
Die Implementierung verwendet Android's \texttt{ConnectivityManager} für die Erkennung von Netzwerkänderungen.

Die Tenant-Konfiguration ermöglicht es, verschiedene Build-Varianten mit unterschiedlichen URLs und Tokens zu verwenden:

\begin{table}[H]
\centering
\begin{tabular}{|c|c|c|c|}
\hline
\textbf{Tenant} & \textbf{Base URL} & \textbf{Client Token} & \textbf{Build Variant} \\
\hline
HSLU I & hslui.mobile-hslu.ch & xxxxxxxx-xxxx-xxxx-xxxx-xxxxxxxxxxxx & hslui \\
\hline
HSLU TA & hsluta.mobile-hslu.ch & yyyyyyy-yyyy-yyyy-yyyy-yyyyyyyyyyyy & hsluta \\
\hline
\end{tabular}
\caption{Tenant-Konfiguration für verschiedene Build-Varianten}
\label{tab:commonnetwork-tenants}
\end{table}

\subsubsection*{Weitere Informationen}

\begin{table}[h]
\centering
\begin{tabularx}{\textwidth}{|l|X|}
\hline
\textbf{Aspekt} & \textbf{Beschreibung} \\
\hline
\textbf{Asynchrone Operationen} & 
Alle Netzwerkoperationen werden asynchron über Kotlin Coroutines ausgeführt. Die Verwendung von suspend-Funktionen stellt sicher, dass alle Operationen nicht-blockierend sind und den Main-Thread nicht beeinträchtigen. 
Dies ermöglicht eine reaktive Programmierung und verbessert die Performance der Anwendung. \\
\hline
\textbf{Type-Safe API} & 
Die API ist type-safe mit Compile-Time-Checks. Methoden wie \texttt{getAsJsonObject()}, \texttt{getAsJsonString()} und \texttt{getAsDownload()} bieten typsichere Rückgabewerte und reduzieren Laufzeitfehler. \\
\hline
\textbf{Multitenant-Unterstützung} & 
Der Multitenant-Ansatz wird durch \texttt{TenantUrlProvider} und tenant-spezifische Implementierungen unterstützt. 
Je nach Build-Variant werden automatisch die korrekten URLs und Tokens verwendet, wodurch Redundanzen verhindert werden. \\
\hline
\textbf{StateFlow für reaktives UI} & 
StateFlow wird für reaktives State-Management verwendet, ähnlich wie \texttt{@Published} in SwiftUI. 
Dies ermöglicht es, UI-Komponenten reaktiv auf Netzwerkstatusänderungen zu reagieren. \\
\hline
\end{tabularx}
\caption{Technische Aspekte des \texttt{CommonNetwork} Moduls}
\label{tab:commonnetwork-specs}
\end{table}

\hspace{1.00cm}