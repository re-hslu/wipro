% ================== Preamble ===================================================

% ------------------------------------------------------------------------------
% Pakete
% ------------------------------------------------------------------------------

\usepackage{scrhack}                                % KOMA-Kompatibilität verbessern
\usepackage{scrlayer-scrpage}                       % Kopf- und Fußzeilen
\usepackage{fontspec}                               % Ermöglicht Systemschriftarten (LuaLaTeX/XeLaTeX)
\usepackage[ngerman]{babel}                         % Deutsche Spracheinstellungen
\usepackage{graphicx, float}                        % Grafiken und Floating-Umgebungen
\usepackage{amsmath, amssymb, amsthm}               % Mathematische Pakete
\usepackage[a4paper, top=2.5cm, bottom=2.5cm, 
            left=1.5cm, right=1.5cm]{geometry}      % Seitenränder
\usepackage{setspace}                               % Zeilenabstand
\usepackage{url}                                    % URLs im Text
\usepackage[hidelinks]{hyperref}                    % Verlinkungen ohne Rahmen
\usepackage{tcolorbox}                              % Farbboxen
\usepackage{tikz}                                   % Grafiken
\usepackage{pgfplots}                               % Diagramme
\usepackage{listings}                               % Quellcode
\usepackage{xcolor}                                 % Farben
\usepackage{soul}                                   % Textmarkierung
\usepackage[backend=biber,style=apa]{biblatex}      % Literaturverzeichnis
\addbibresource{literatur.bib}                      % Literaturverzeichnis
\DeclareLanguageMapping{ngerman}{ngerman-apa}       % für deutsche APA-Formatierung
\usepackage{csquotes}                               % Korrekte Zitate
\usepackage{acronym}                                % Abkürzungsverzeichnis
\usepackage{array}                                  % Erweiterte Tabellen
\usepackage{csvsimple}
\usepackage{multicol}                               % Mehrspaltiger Text
\usepackage{pdfpages}                               % PDF-Einbindung
\usepackage{enumitem}                               % Erweiterte Listensteuerung
\usepackage{pdflscape}                              % Querformatseiten
\usepackage{dirtree}                                % Verzeichnisstruktur darstellen       
\usepackage{pdfpages}                               % PDFs einbinden
\usepackage{graphicx}                               % für Bilder (PNG/JPG)
\usepackage{tabularx}
\usepackage{booktabs}
\usepackage{array}
\usepackage{caption}

\usetikzlibrary{trees, mindmap}                     % TikZ-Erweiterungen
\graphicspath{{Fotos/}}                             % Standardpfad für Bilder
\pgfplotsset{compat=1.18}

% Abbildungsnummerierung global fortlaufend
\usepackage{chngcntr}
\counterwithout{figure}{chapter}

% ------------------------------------------------------------------------------
% Schriftart (Calibri)
% ------------------------------------------------------------------------------
\setmainfont{Calibri}
\setsansfont{Calibri}
\setmonofont{Consolas}

% ------------------------------------------------------------------------------
% Zeilenabstand und Absatzformatierung
% ------------------------------------------------------------------------------
\setstretch{1.12}
\setlength{\parskip}{0.8em}
\setlength{\parindent}{0pt}

% ------------------------------------------------------------------------------
% Listen-Abstände anpassen (kein zusätzlicher Abstand zwischen Items)
% ------------------------------------------------------------------------------
\setlist[itemize]{noitemsep, topsep=0pt, partopsep=0pt, parsep=0pt}
\setlist[enumerate]{noitemsep, topsep=0pt, partopsep=0pt, parsep=0pt}

% ------------------------------------------------------------------------------
% Kapitelüberschriften (Abstand oben/unten)
% ------------------------------------------------------------------------------
\RedeclareSectionCommand[
  beforeskip=-15pt,   % Negativ = näher am oberen Seitenrand
  afterskip=12pt
]{chapter}

% ------------------------------------------------------------------------------
% Kopf- und Fußzeilen
% ------------------------------------------------------------------------------
\pagestyle{scrheadings}

\ihead{HSLU Mobile Apps - Android Jetpack Compose - AI First}
\ohead{WIPRO HS25}
\cfoot{}
\ofoot{\thepage}

\KOMAoptions{
  headsepline = 0.4pt:\textwidth,
  footsepline = 0.4pt:\textwidth
}

\renewcommand*{\chapterpagestyle}{scrheadings}

% ------------------------------------------------------------------------------
% Codeformatierung
% ------------------------------------------------------------------------------
\definecolor{codegreen}{rgb}{0,0.8,0}
\definecolor{codegray}{rgb}{0.3,0.3,0.3}
\definecolor{codeorange}{rgb}{0.8,0.3,0}
\definecolor{codeblue}{rgb}{0,0,0.8}
\definecolor{backcolour}{rgb}{0.95,0.95,0.96}

\lstdefinestyle{mystyle}{
    backgroundcolor     =   \color{backcolour},   
    commentstyle        =   \color{codegreen},
    keywordstyle        =   \color{codeorange},
    numberstyle         =   \tiny\color{codegray},
    stringstyle         =   \color{codeblue},
    basicstyle          =   \ttfamily\footnotesize,
    breaklines          =   true,   
    captionpos          =   b,                    
    keepspaces          =   true,                 
    numbers             =   left,                    
    numbersep           =   5pt,                  
    showspaces          =   false,                
    showstringspaces    =   false,
    showtabs            =   false,                  
    tabsize             =   2,
    xleftmargin         =   10pt,
}
\lstset{style=mystyle}

% ------------------------------------------------------------------------------
% Seitennummerierung (nur arabisch, keine römischen Zahlen)
% ------------------------------------------------------------------------------
\AtBeginDocument{
  \pagenumbering{arabic}
  \setcounter{page}{1}
}

% Bis 4 Stellen nummerieren
\setcounter{secnumdepth}{4}

% Kein Punkt nach der Abbildungsnummer
\usepackage{caption}
\captionsetup[figure]{labelsep=colon}
\renewcommand{\figurename}{Abbildung}