% ==============================================================================
% Autor Vorlage:    Amir Suter
% Datum:            20.09.2024
% 
% Hochschule Luzern T&A
%
% ==============================================================================

\usepackage{scrhack}                                % Hinzufügen, um bestimmte Warnungen zu vermeiden
\usepackage{scrlayer-scrpage}                       % Für Kopf- und Fusszeilen bei Verwendung von KOMA-Script
\usepackage[utf8]{inputenc}                         % Verwendung von UTF-8 Codierung
\usepackage[T1]{fontenc}                            % Für korrekte Silbentrennung und Schriftzeichen
\usepackage[ngerman]{babel}                         % Korrekte Formatierung und Übersetzungen für Deutsch               
\usepackage{graphicx, float}                        % Paket zum Einfügen und Anpassen von Grafiken
\usepackage{amsmath, amssymb, amsthm}               % Mathematische Umgebungen und Symbole
\usepackage[a4paper, top=2.5cm, bottom=2.5cm, 
            left=1.5cm, right=1.5cm]{geometry}      % Seitenränder für A4-Papier einstellen
\usepackage{setspace}                               % Paket zur Anpassung des Zeilenabstands
\usepackage{url}                                    % Für die Darstellung von URLs
\usepackage[hidelinks]{hyperref}                    % Hyperlinks im Dokument, ohne farbige Rahmen
\usepackage{tcolorbox}                              % Für farbige Boxen zur besseren Darstellung von Inhalten
\usepackage{tikz}                                   % Für das Erstellen von Grafiken
\usepackage{pgfplots}                               % Für das Erstellen von Plots und Diagrammen
\usepackage{listings}                               % Für die Darstellung von Quellcode
\usepackage{xcolor}                                 % Für die Definition eigener Farben
\usepackage{soul}                                   % Für Textmarkierung
\usepackage[backend=biber, style=numeric]{biblatex}  % für Zitate und Literaturverzeichnis      citestyle=apa, bibstyle=apa
\usepackage{csquotes}                               % Für korrekte Zitate in verschiedenen Sprachen
\usepackage{acronym}                                % Für Abkürzungsverzeichnis und -definitionen
\usepackage{array}                                  % Für erweiterte Tabellenformatierung
\usepackage{csvsimple}
\usepackage{multicol}                               % Für mehrspaltige Texte oder Auflistungen
\usepackage{hyperref}                               % Für Hyperlinks
\usepackage{pdfpages}                               % PDFs in den Anhang (oder auch sonst wo)
\usepackage{enumitem}                               % Erforderlich für erweiterte Listen, z.B. alphabetisch
\usepackage{pdflscape}                              % Für Querformat
\usepackage{graphicx}                               % Für Grafiken und Bilder
\usepackage{enumitem}                               % Für erweiterte Listen 



\addbibresource{referenzen.bib}                     % Einbindung der Bibliografie-Ressource

\usetikzlibrary{trees, mindmap}                     % TikZ-Bibliotheken für Baumstrukturen und Mindmaps

\graphicspath{{Fotos/}}                             % Pfad für das Laden von Grafiken festlegen

\pgfplotsset{compat=1.18}                           % Verwendung der neuesten Version von PGFPlots


% ------------ Kopf-und Fusszeile -----------------------------------------------

\pagestyle{scrheadings}

\ihead{Dokumentation PREN1}
\chead{}
\ohead{Gruppe 21}

\ifoot{HSLU Technik und Architektur, HSLU Informatik}
\cfoot{}
\ofoot{\thepage}

\KOMAoptions{
    headsepline = 0.4pt:\textwidth,  % Linie unter der Kopfzeile (Dicke 0.4pt, Länge 90% der Seitenbreite)
    footsepline = 0.4pt:\textwidth   % Linie über der Fusszeile (Dicke 0.4pt, Länge 90% der Seitenbreite)
}

\renewcommand*{\chapterpagestyle}{scrheadings}  % fügt auch bei chapter Kopf- und Fusszeilen hinzu

% ------------ Code Formatierung -----------------------------------------------

\definecolor{codegreen}{rgb}{0,0.8,0}
\definecolor{codegray}{rgb}{0.3,0.3,0.3}
\definecolor{codeorange}{rgb}{0.8,0.3,0}
\definecolor{codeblue}{rgb}{0,0,0.8}
\definecolor{backcolour}{rgb}{0.95,0.95,0.96}

\lstdefinestyle{mystyle}{
    backgroundcolor     =   \color{backcolour},   
    commentstyle        =   \color{codegreen},
    keywordstyle        =   \color{codeorange},
    numberstyle         =   \tiny\color{codegray},
    stringstyle         =   \color{codeblue},
    basicstyle          =   \ttfamily\footnotesize,
    breakatwhitespace   =   false,         
    breaklines          =   true,   
    mathescape          =   true,              
    captionpos          =   b,                    
    keepspaces          =   true,                 
    numbers             =   left,                    
    numbersep           =   5pt,                  
    showspaces          =   false,                
    showstringspaces    =   false,
    showtabs            =   false,                  
    tabsize             =   2,
    xleftmargin         =   10pt,
}
\lstset{style=mystyle}


% ------------ eingene Farben --------------------------------------------------

\definecolor{PRENpurple} {rgb} {0.37, 0.16, 0.81}
\definecolor{PRENmint}   {rgb} {0.62, 0.88, 0.75}


% ------------ Colorbox Einstellungen ------------------------------------------

\tcbset{colback=PRENpurple!30!white, colframe=PRENpurple, boxsep=5pt, arc=4pt, auto outer arc, width=\linewidth, left=5pt, right=5pt, top=0pt, bottom=0pt}


% ------------ Text-Formatierungen ---------------------------------------------

\setstretch{1.12}               % Zeilenabstand

\setlength{\parskip}{0.8em}     % legt den vertikalen Abstand zwischen aufeinanderfolgenden Absätzen fest
\setlength{\parindent}{0pt}     % legt die Einrückung der ersten Zeile eines neuen Absatzes fest

\RedeclareSectionCommand[
  beforeskip=0pt, % Abstand vor \chapter
  afterskip=12pt  % Abstand nach \chapter
]{chapter}