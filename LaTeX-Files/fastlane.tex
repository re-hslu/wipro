\subsection{Fastlane}

Das Fastlane-Setup für die Android-App automatisiert den Build- und Deployment-Prozess für beide Tenants (HSLU I und HSLU TA). 
Anstatt separate Fastfiles pro Tenant zu verwenden wie das im Android-XML Projekt der Fall ist, wurde ein einheitliches, parametrisiertes Fastfile implementiert, 
das Redundanzen eliminiert und die Wartbarkeit verbessert. Das Fastfile unterstützt Debug- und Release-Builds sowie automatische Uploads zu Google Play Store in verschiedenen Tracks (Beta, Internal, Production).

\subsubsection*{Ziel und Motivation}

Das Hauptziel besteht darin, die CI/CD-Pipeline zu vereinfachen und Redundanzen zu eliminieren. 
Durch die Parametrisierung mit dem \texttt{tenant}-Parameter kann ein einziges Fastfile für beide Tenants verwendet werden, 
was die Wartbarkeit erheblich verbessert. Zusätzlich werden alle sensiblen Daten (Keystores, API-Keys) sicher in einem separaten Git-Repository gespeichert und verschlüsselt, 
sodass sie nicht im Haupt-Repository liegen.

Ein weiteres wichtiges Ziel ist die Automatisierung des gesamten Build- und Deployment-Prozesses, 
von der Kompilierung über die Signierung bis hin zum Upload in den Google Play Store. 
Dies reduziert manuelle Fehler und beschleunigt den Release-Prozess erheblich.

\subsubsection*{Umsetzung / Funktionsweise}

Die Implementierung basiert auf einem zentralen \texttt{before\_all}-Block, der die Initialisierung und Konfiguration für alle Lanes übernimmt. 
Das Fastfile verwendet Dotenv für die Verwaltung von Secrets und parametrisiert alle tenant-spezifischen Werte. 
Der \texttt{before\_all}-Block wird vor jeder Lane ausgeführt und übernimmt die Validierung des \texttt{tenant}-Parameters, 
die Definition von Pfaden sowie das Mapping von Tenant-Namen zu Android-Modulen und Package-IDs.

Die tenant-spezifischen Umgebungsvariablen (Keystore-Passwörter, API-Keys, etc.) werden aus \texttt{.env.secret} geladen 
und in generische Variablen gemappt, sodass alle Lanes die gleichen Variablennamen verwenden können.

\paragraph*{Lanes}

Das Fastfile definiert vier Haupt-Lanes, die in der folgenden Tabelle beschrieben sind:

\begin{table}[h]
\centering
\begin{tabularx}{\textwidth}{|l|X|}
\hline
\textbf{Lane} & \textbf{Beschreibung} \\
\hline
\texttt{build\_debug} & Erstellt einen Debug-Build ohne Signierung. Führt \texttt{gradle clean assemble} für das entsprechende Tenant-Modul aus und kopiert die Artefakte in den Output-Ordner. Wird hauptsächlich für lokale Tests verwendet. \\
\hline
\texttt{build\_release} & Erstellt einen signierten Release-Build. Lädt den verschlüsselten Keystore aus dem Zertifikats-Repository, entschlüsselt ihn, erstellt die \texttt{keystore.properties}-Datei für Gradle und führt \texttt{gradle clean bundle} aus. Das signierte AAB wird in den Output-Ordner kopiert. \\
\hline
\texttt{beta} & Baut ein signiertes Release-Bundle und lädt es in den Google Play Store hoch. Standardmässig wird der Beta-Track verwendet, kann aber über den \texttt{track:}-Parameter überschrieben werden (z.\,B. \texttt{track:internal}). Lädt sowohl Keystore als auch Google Play API-Key aus dem Zertifikats-Repository. Ruft intern \texttt{build\_release} auf und lädt anschliessend das AAB hoch. \\
\hline
\texttt{release} & Analog zu \texttt{beta}, lädt jedoch explizit in den Production-Track des Google Play Stores. Wird für finale Releases verwendet. \\
\hline
\end{tabularx}
\caption{Übersicht der Fastlane-Lanes}
\label{tab:fastlane-lanes}
\end{table}

\paragraph*{Hilfsfunktionen}

Das Fastfile verwendet mehrere Hilfsfunktionen zur Verwaltung von Zertifikaten und Konfiguration:

\begin{itemize}
    \item \texttt{download\_from\_certs\_repo}: Klont das Zertifikats-Repository und entschlüsselt eine Datei (typischerweise den Keystore) mit OpenSSL (AES-256-CBC).
    \item \texttt{download\_from\_certs\_repo\_full}: Erweiterte Variante, die sowohl Keystore als auch Google Play API-Key entschlüsselt.
    \item \texttt{create\_keystore\_properties}: Erstellt die \texttt{keystore.properties}-Datei dynamisch, die von Gradle für die Signierung verwendet wird.
    \item \texttt{clean\_directory}: Entfernt temporäre Zertifikatsdateien nach dem Build, um Sicherheitsrisiken zu minimieren.
\end{itemize}

\subsubsection*{Verwendung und Sicherheit}

Die Verwendung erfolgt durch Aufruf von Fastlane mit dem \texttt{tenant}-Parameter: 
\texttt{fastlane <lane> tenant:hslui} oder \texttt{fastlane <lane> tenant:hsluta}. 
Ohne diesen Parameter bricht Fastlane mit einer Fehlermeldung ab.

Alle sensiblen Daten (Keystores, API-Keys) werden in einem separaten Git-Repository gespeichert und mit OpenSSL verschlüsselt (AES-256-CBC). 
Die Passwörter für die Entschlüsselung werden aus \texttt{.env.secret} geladen, das nicht im Repository gespeichert wird, 
sondern von der CI/CD-Pipeline zur Laufzeit erstellt wird. Die Keystore-Properties-Datei wird dynamisch erstellt 
und nach dem Build gelöscht, um Sicherheitsrisiken zu minimieren.

Die Upload-Funktionalität unterstützt verschiedene Tracks im Google Play Store. 
Der Standard-Track für \texttt{beta} ist "beta", kann aber über den \texttt{track:}-Parameter überschrieben werden (z.\,B. \texttt{track:internal}). 
Changelogs werden bewusst übersprungen, da diese manuell im Google Play Console verwaltet werden.

Das Fastfile ist erweiterbar für weitere Tenants, indem einfach ein neuer \texttt{when}-Fall im 
\texttt{before\_all}-Block hinzugefügt wird und die entsprechenden Umgebungsvariablen in \texttt{.env.secret} definiert werden.
