Das Backend stellt dabei die zentrale Schnittstelle für alle mobilen Applikationen dar
und sorgt dafür, dass Inhalte plattformunabhängig bereitgestellt werden können.
Module wie News, Blog, Mensa oder WebLinks werden serverseitig verwaltet und den
Apps über definierte APIs zur Verfügung gestellt.

Durch diesen Ansatz können neue Inhalte oder Anpassungen vorgenommen werden,
ohne dass eine neue Version der App im App Store oder Play Store veröffentlicht
werden muss. Dies vereinfacht den Betrieb und reduziert den Wartungsaufwand
auf Seiten der mobilen Applikationen deutlich.

Zudem ermöglicht das gemeinsame Backend eine einheitliche Datenbasis für iOS
und Android, wodurch Unterschiede im Funktionsumfang oder im angezeigten Inhalt
vermieden werden können.

Für den Betrieb der Applikationen existieren mehrere Backend-Umgebungen.
Neben dem produktiven Backend für den regulären Betrieb gibt es separate
Backends für die Departemente Informatik (I) sowie Technik \& Architektur (TA).
Zusätzlich steht eine eigene QA- bzw. Entwicklungsumgebung zur Verfügung,
welche für Tests, neue Features und technische Anpassungen genutzt wird.

\vspace{1.2em}
\begin{figure}[H]
    \centering
    \includegraphics[width=0.99\textwidth]{Fotos/backend.png}
    \caption{Screenshot des QA-Backends}
    \label{fig:backendimage}
\end{figure}