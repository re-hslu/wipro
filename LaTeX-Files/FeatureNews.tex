\subsection{News} \label{sec:featurenews}

Das Feature \textit{FeatureNews} dient dazu, die aktuellen Neuigkeiten der HSLU über
die WordPress-basierte Plattform \url{https://news.hslu.ch/} direkt in der App anzuzeigen.
Ursprünglich war vorgesehen, die Inhalte über eine einfache WebView zu laden, doch im
Laufe des Semesters wurde die Lösung vollständig auf eine JSON-basierte Darstellung
umgestellt.

\subsection*{Motivation und Zielsetzung}

Im ersten Schritt wird der anzuzeigende News-Link nicht hart im Code definiert, sondern
von der API geladen. Dadurch können die App-Verantwortlichen die Quelle jederzeit
anpassen, ohne dass die App neu kompiliert werden muss. Zudem ermöglicht dies eine
saubere Trennung für unterschiedliche Tenants (z.\,B. \textit{HSLU-I} und \textit{HSLU-TA}),
da beide auf verschiedene APIs zugreifen und somit unterschiedliche News-Quellen
verwenden können.

Die ursprüngliche Umsetzung im früheren XML-basierten Android-Projekt verwendete
die Accompanist-WebView \parencite{noauthor_accompanist_nodate}. Da diese Library jedoch als \textit{deprecated} markiert
wurde, ersetzten wir sie durch die native \texttt{android.webkit.WebView}, welche
alle benötigten Funktionen bereitstellt.

Während des Semesters fiel jedoch dann die Entscheidung, auf beiden Plattformen (iOS und
Android) auf WebViews zu verzichten. Gründe dafür waren insbesondere
Datenschutzaspekte, fehlende Kontrolle über Tracking-Mechanismen externer Websites
sowie die Tatsache, dass WebViews Abhängigkeiten zu fremden Cookies, Skripten und
Datenschutzrichtlinien erzeugen.

Die Lösung wurde deshalb auf ein JSON-basiertes Rendering umgestellt: Die App lädt
nicht mehr die komplette Website, sondern nur die strukturierten Daten des Artikels
und rendert den Inhalt vollständig nativ.

\subsection*{Funktionsweise JSON-View}

Die News werden über die WordPress-REST-API von \texttt{news.hslu.ch} als JSON geladen.
Der Download erfolgt über den \texttt{NewsDataLoader}, der auf der gemeinsamen Utility-Klasse
\texttt{JsonDataLoader} basiert. Diese stellt ein hybrides Ladeverhalten bereit:  
\begin{itemize}
  \item Bei aktiver Internetverbindung werden die Daten remote geladen und lokal
        gecached.
  \item Ohne Internet werden die News aus dem Cache gelesen (falls vorhanden).
\end{itemize}

Die JSON-Daten werden anschliessend in das Domain-Objekt
\texttt{NewsPostWordpress.kt} konvertiert, das Titel, Beschreibung, Inhalt,
Publikationsdatum, Autor sowie das \textit{featured image} enthält.

Um die Inhalte darzustellen, werden die HTML-Fragmente der WordPress-API mit einem
HTML-Parser verarbeitet. Der Parser extrahiert Absätze, Überschriften, Blockquotes
und wandelt HTML-Entities wie \texttt{\&nbsp;} oder \texttt{\&ldquo;} in reguläre Zeichen um.
Der Parser erzeugt daraus eine strukturierte Liste von Content-Blöcken (z.\,B.
\texttt{Subheading}, \texttt{Paragraph}), die anschliessend vom Compose-Renderer
(\texttt{NewsContentRenderer}) visuell aufbereitet werden. Die grundlegende Logik des
Parsers basiert auf regulären Ausdrücken und Entity-Decoding.

\newpage

\subsection*{Darstellung der News}

Die Darstellung erfolgt dann vollständig in Jetpack Compose und besteht aus zwei Bereichen:

\begin{itemize}
  \item \textbf{NewsOverviewWordpressView}:  
        Zeigt eine Übersicht aller verfügbaren News. Jede News wird als Karte mit
        Titel, Bild, Kurzbeschreibung und Datum dargestellt.  
        Die View beobachtet den Ladezustand des \newline \texttt{NewsDataLoader} und reagiert
        reaktiv auf Statusänderungen.
  \item \textbf{NewsDetailWordpressView}:  
        Öffnet einen einzelnen Artikel als Dialog oder Vollbildansicht.  
        Die View rendert den gesamten strukturierten Inhalt, erlaubt das Ausklappen
        langer Texte und bietet einen Button \textit{„Im Browser öffnen“}, falls der
        Nutzer den Originalartikel betrachten möchte.
\end{itemize}

\subsection*{Herausforderungen und Einschränkungen}

Durch den Wechsel von der WebView zur nativen JSON-Darstellung ist die App nun
darauf angewiesen, dass die HTML-Struktur der WordPress-Beiträge stabil bleibt.
Während der Parser viele HTML-Entitäten und Formatierungen zuverlässig erkennt,
ist es technisch nicht möglich, alle denkbaren Formatierungsvarianten eindeutig
zu interpretieren.  

Beispielsweise kann ein Redaktor einen Titel visuell fett und gross darstellen,
ohne ihn als \texttt{<h1>}–\texttt{<h6>} zu markieren. In der WebView wäre dies
optisch korrekt, aber unser Parser kann diese Formatierung nicht als Titel
erkennen und behandelt sie als normalen Text. Solche Abweichungen liegen ausserhalb
des Einflussbereichs der App und sind nur durch redaktionelle Disziplin vermeidbar.

Falls WordPress ein Update einführt oder die API plötzlich andere Strukturen liefert,
kann dies zu Darstellungsproblemen führen. Im Notfall könnte das Backend temporär
wieder auf die alte WebView-Variante zurückschalten, bis das neue Format korrekt
unterstützt wird.

\vspace{0.5cm}

\begin{figure}[H]
    \centering
    \begin{subfigure}[b]{0.23\textwidth}
        \centering
        \includegraphics[width=\textwidth]{Fotos/feature-screenshots/news_web.png}
        \caption{Web-Ansicht}
        \label{fig:news_web}
    \end{subfigure}
    \hspace{0.3cm}
    \begin{subfigure}[b]{0.23\textwidth}
        \centering
        \includegraphics[width=\textwidth]{Fotos/feature-screenshots/news_json.png}
        \caption{JSON-Übersicht}
        \label{fig:news_overview}
    \end{subfigure}
    \hspace{0.3cm}
    \begin{subfigure}[b]{0.23\textwidth}
        \centering
        \includegraphics[width=\textwidth]{Fotos/feature-screenshots/news_json_detail.png}
        \caption{JSON-Detailansicht}
        \label{fig:news_detail}
    \end{subfigure}
    \caption{Screenshots des News-Features}
    \label{fig:news_feature}
\end{figure}
