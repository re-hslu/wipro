\subsection{Info-, Einstellungen- und WebLinks-Features}

Die Features \textit{Info}, \textit{Einstellungen} und \textit{WebLinks} übernehmen in der App kleinere Funktionen. 
Da sich diese Features in Aufbau und Darstellung stark ähneln, wurde ein grosser Teil des Layouts 
zentral im \texttt{common}-Modul umgesetzt und wiederverwendet,
um Code-Duplikationen zu vermeiden und eine einheitliche Darstellung sicherzustellen.

\subsubsection*{WebLinks}
Das Feature \textit{WebLinks} stellt eine Liste von weiterführenden Links zur Verfügung, die auf externe Seiten
der HSLU verweisen. Inhalte, die nicht direkt in der App umgesetzt sind, können so trotzdem einfach erreicht
werden. Die Liste der Links wird über die Backend-API geladen, lokal zwischengespeichert und in einer
Liste dargestellt.  
Beim Antippen eines Eintrags wird der jeweilige Link im externen Standard-Browser des Geräts geöffnet.
Das Feature unterstützt Deutsch und Englisch, wobei die Sprache automatisch anhand der Geräteeinstellung
gewählt wird. Falls eine Übersetzung nicht verfügbar ist, wird jeweils auf die andere Sprache zurückgegriffen
(siehe~\ref{sec:localisation}).

\subsubsection*{Info}
Das Feature \textit{Info} (bzw. \textit{About}) zeigt allgemeine Informationen zur App an. Dazu gehören unter anderem
der Name des Tenants, die aktuell installierte App-Version, Kontaktinformationen, ein Blog-Link sowie eine Liste
der bisherigen Mitwirkenden. Zusätzlich werden rechtliche Hinweise wie Disclaimer oder Copyright angezeigt.
Diese Informationen werden ebenfalls über die Backend-API geladen. Inhalte werden nur angezeigt, wenn sie
tatsächlich vorhanden sind, leere Felder bleiben ausgeblendet.

\subsubsection*{Einstellungen}
Das Feature \textit{Einstellungen} dient der technischen Verwaltung der App-Daten. Hier können gecachte
Inhalte einzelner Module (z.\,B. News, Blog, Mensa oder WebLinks) manuell neu geladen oder zurückgesetzt werden.
Dies ist bei Problemen mit veralteten Daten hilfreich.  
Zusätzlich enthält das Settings-Feature die Konfiguration für den Stundenplan, bei der der verwendete
Kalender ausgewählt werden kann.

\vspace{0.5cm}

\begin{figure}[H]
    \centering

    \begin{subfigure}[b]{0.23\textwidth}
        \centering
        \includegraphics[width=\textwidth]{Fotos/feature-screenshots/links.png}
        \caption{Links}
        \label{fig:links}
    \end{subfigure}
    \hspace{0.3cm}
    \begin{subfigure}[b]{0.23\textwidth}
        \centering
        \includegraphics[width=\textwidth]{Fotos/feature-screenshots/about.png}
        \caption{Info/About}
        \label{fig:about}
    \end{subfigure}
    \hspace{0.3cm}
    \begin{subfigure}[b]{0.23\textwidth}
        \centering
        \includegraphics[width=\textwidth]{Fotos/feature-screenshots/settings.png}
        \caption{Einstellungen}
        \label{fig:settings}
    \end{subfigure}
    \caption{Screenshots der Features}
    \label{fig:3_features}
\end{figure}