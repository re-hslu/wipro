\subsection{CommonUtil}

Das Modul \texttt{CommonUtil} stellt zentrale Hilfsfunktionen bereit, die von
allen Feature-Modulen genutzt werden. Dazu gehören unter anderem
Datumsformatierungen, das Parsen von HTML-Inhalten sowie das Laden und
Verarbeiten von JSON-Daten. Durch diese zentrale Sammlung werden häufig
verwendete Operationen gebündelt und Code-Duplikationen vermieden.

Ziel des Moduls ist es, eine einheitliche und konsistente Nutzung solcher
Hilfsfunktionen über die gesamte Applikation hinweg sicherzustellen.
Komplexere Aufgaben, wie das Extrahieren von Inhalten aus HTML oder das
Verarbeiten strukturierter Daten, werden an einer zentralen Stelle
implementiert. Dadurch können Anpassungen oder Fehlerkorrekturen effizient
vorgenommen werden und wirken sich automatisch auf alle betroffenen Features
aus.

\subsubsection*{Umsetzung / Funktionsweise}

Das Modul besteht aktuell aus drei Hauptkomponenten: \texttt{DateFormatter} für Datumsformatierung, \newline \texttt{HtmlContentParser} für HTML-Parsing und 
\texttt{JsonDataLoader} für JSON-Datenlade-Operationen.

Die \texttt{DateFormatter} Klasse bietet Funktionen zur Formatierung von ISO-Datumsstrings in deutsches Format. 
Die \texttt{formatDate()}-Methode konvertiert ein ISO-formatierte Datum (yyyy-MM-dd'T'HH:mm:ss) in deutsches Format (dd.MM.yyyy):

\begin{lstlisting}[language=Kotlin, caption={DateFormatter}]
object DateFormatter {
    fun formatDate(isoDate: String): String {
        val inputFormat = SimpleDateFormat("yyyy-MM-dd'T'HH:mm:ss", Locale.US)
        val outputFormat = SimpleDateFormat("dd.MM.yyyy", Locale("de", "CH"))
        return outputFormat.format(inputFormat.parse(isoDate) ?: Date())
    }
}
\end{lstlisting}

Die \texttt{HtmlContentParser} bietet Funktionen zum Parsen von HTML-Content und Extraktion von Content-Blöcken. 
Dies wird unter anderem bei den Features News und Blog verwendet, um Daten aus der WordPress-Api in saubere Composable-Elemete zu konvertieren.
Die \texttt{stripOutHtml()}-Extension-Funktion entfernt HTML-Tags und dekodiert HTML-Entities:

\begin{lstlisting}[language=Kotlin, caption={Beispiel HTML-Tags dekodieren}]
fun String.stripOutHtml(): String {
    return this.replace(Regex("<[^>]*>"), "")
        .replace("&nbsp;", " ")
        .replace("&ldquo;", "\"")
        // ...
}
\end{lstlisting}

Die \texttt{parseContentBlocks()}-Funktion extrahiert strukturierte Content-Blöcke aus HTML und erzeugt eine Liste von \texttt{ContentBlock}-Objekten (z.\,B. \texttt{Paragraph}, \texttt{Subheading}, \texttt{Blockquote}).

Die \texttt{JsonDataLoader} Klasse bietet generische Funktionen für JSON-Datenlade-Operationen. 
Die \texttt{syncData()}-Funktion lädt Daten von Remote oder aus dem lokalen Cache und implementiert eine Offline-First-Strategie.
Die \texttt{reset()}-Funktion ermöglicht das Zurücksetzen und Neu-Synchronisieren von Daten, indem der lokale Cache gelöscht und die Daten erneut vom Remote-Server geladen werden.